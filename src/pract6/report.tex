\graphicspath{{./img}}

\section*{\LARGE{Введение}}
\addcontentsline{toc}{section}{Введение}
\textbf{Цель}: научиться работать с ресурсами в Android Studio.\par
Ресурс в приложении Android представляет собой файл, например, файл
разметки интерфейса или некоторое значение, например, простую строку. То
есть ресурсы представляют собой и файлы разметки, и отдельные строки, и
звуковые файлы, файлы изображений и т.д. Все ресурсы находятся в проекте
в каталоге res. Для различных типов ресурсов, определенных в проекте, в
каталоге res создаются подкаталоги. Поддерживаемые подкаталоги:

\begin{itemize}
	\item animator/: xml-файлы, определяющие анимацию свойств;
	\item anim/: xml-файлы, определяющие tween-анимацию;
	\item color/: xml-файлы, определяющие список цветов;
	\item drawable/: Графические файлы (.png, .jpg, .gif);
	\item mipmap/: Графические файлы, используемые для иконок приложения;
		под различные разрешения экранов;
	\item layout/: xml-файлы, определяющие пользовательский интерфейс
		приложения;
	\item menu/: xml-файлы, определяющие меню приложения;
	\item raw/: различные файлы, которые сохраняются в исходном виде;
	\item values/: xml-файлы, которые содержат различные используемые в
		приложении значения, например, ресурсы строк;
	\item xml/: Произвольные xml-файлы;
	\item font/: файлы с определениями шрифтом и расширениями .ttf, .otf или
		.ttc, либо файлы XML, который содержат элемент <font-family>.
\end{itemize}

Существует два способа доступа к ресурсам: в файле исходного кода и в
файле xml.
Ссылка на ресурсы в коде
Тип ресурса в данной записи ссылается на одно из пространств (вложенных
классов), определенных в файле R.java, которые имеют соответствующие им
типы в xml:

\begin{itemize}
	\item R.drawable (ему соответствует тип в xml drawable);
	\item R.id (id);
	\item R.layout (layout);
	\item R.string (string);
	\item R.attr (attr);
	\item R.plural (plurals);
	\item R.array (string-array).
\end{itemize}

Нередко возникает необходимость ссылаться на ресурс в файле xml,
например, в файле, который определяет визуальный интерфейс, к примеру, в
activity\_main.xml. Ссылки на ресурсы в файлах xml имеют следующую
формализованную форму: \verb|@[имя_пакета:]тип_ресурса/имя_ресурса|. Где:
\begin{itemize}
	\item имя\_пакета представляет имя пакета, в котором ресурс находится
		(указывать необязательно, если ресурс находится в том же пакете);
	\item тип\_ресурса представляет подкласс, определенный в классе R для
		типа ресурса;
	\item имя\_ресурса имя файла ресурса без расширения или значение
		атрибута android:name в XML-элементе (для простых значений).
\end{itemize}

\clearpage

\section*{\LARGE{Выполнение практической работы}}
\addcontentsline{toc}{section}{Выполнение практической работы}

\section{Ресурсы строк}
Ресурсы строк --- один из важных компонентов приложения. Мы используем
их при выведении названия приложения, различного текста, например, текста
кнопок и т.д.\par
XML-файлы, представляющие собой ресурсы строк, находятся в проекте в
каталоге res/values. По умолчанию ресурсы строк находятся в файле strings.xml,
который может выглядеть следующим образом.\par
\begin{lstlisting}[language=xml, caption=\leftline{Использование ресурсов строк в XML-коде}, label=lst:xml:string]
<TextView
        android:id="@+id/textView"
        android:layout_width="wrap_content"
        android:layout_height="wrap_content"
        android:text="defalt"
        app:layout_constraintBottom_toBottomOf="parent"
        app:layout_constraintEnd_toEndOf="parent"
        app:layout_constraintStart_toStartOf="parent"
        app:layout_constraintTop_toTopOf="parent"
        app:layout_constraintVertical_bias="0.1" />
\end{lstlisting}

\begin{lstlisting}[language=Kotlin, caption=\leftline{Использование ресурсов строк в коде}, label=lst:string]
override fun onCreate(savedInstanceState: Bundle?) {
        super.onCreate(savedInstanceState)
        setContentView(R.layout.activity_work61)

        val textView = findViewById<TextView>(R.id.textView)
        textView.setText(R.string.text61)
    }
\end{lstlisting}

\img{img1}{Использование ресурсов строк}

\section{Форматирование строк}
Android позволяет применять к ресурсам строк форматирование.

\begin{lstlisting}[language=xml, caption=\leftline{Форматная строка в каталоге ресурсов}, label=lst:xml:format]
<TextView
        android:id="@+id/textView"
        android:layout_width="wrap_content"
        android:layout_height="wrap_content"
        android:text="@string/text62"
        app:layout_constraintBottom_toBottomOf="parent"
        app:layout_constraintEnd_toEndOf="parent"
        app:layout_constraintStart_toStartOf="parent"
        app:layout_constraintTop_toTopOf="parent"
        app:layout_constraintVertical_bias="0.1" />
\end{lstlisting}

\begin{lstlisting}[language=Kotlin, caption=\leftline{Форматная строка в каталоге ресурсов}, label=lst:kot:format]
override fun onCreate(savedInstanceState: Bundle?) {
        super.onCreate(savedInstanceState)
        setContentView(R.layout.activity_work62)

        val str = getString(R.string.text62, "Kotlin")

        val textView = findViewById<TextView>(R.id.textView)
        textView.text = str
    }
\end{lstlisting}

\img{img2}{Форматная строка в каталоге ресурсов}

\section{Ресурсы Plurals}
Plurals представляют еще один вид набора строк.
Ресурс plurals используется, напрмер если нужно использовать одно
существительное с разными окончаниями, изменяемое в зависимости от
числительного, которое с ним употребляется.\par
Пример файла ресурсов изображен на рисунке~\ref{fig:xml:plurals}.

\begin{lstlisting}[language=xml, caption=\leftline{Объявление ресурсов Plurals}, label=lst:xml:plurals]
<plurals name="flowers" translatable="false">
        <item quantity="one">%d цветок</item>
        <item quantity="few">%d цветка</item>
        <item quantity="many">%d цветков</item>
    </plurals>
\end{lstlisting}

\img{img3}{Ресурсы Plurals}

Для задания ресурса используется элемент \texttt{<plurals>}, для которого
существует атрибут \texttt{name}, получающий в качестве значения произвольное
название, по которому потом ссылаются на данный ресурс.\par
Сами наборы строк вводятся дочерними элементами \texttt{<item>}. Этот элемент
имеет атрибут \texttt{quantity}, который имеет значение, указывающее,
когда эта строка используется. Данный атрибут может принимать следующие
значения:

\begin{itemize}
	\item zero: строка для количества в размере 0;
	\item one: строка для количества в размере 1 (для русского языка - для
		задания всех количеств, оканчивающихся на 1, кроме 11);
	\item two: строка для количества в размере 2;
	\item few: строка для небольшого количества.
\end{itemize}

Причем в данном случае многое зависит от конкретного языка. А система
сама позволяет определить, какое значение брать для того или иного числа.

С помощью метода \texttt{getQuantityString} можно получить значение ресурса.
Первым параметром передаем идентификатор ресурса. Вторым параметром
идет значение. для которого нужно найти нужную строку. Третий параметр
представляет собой значение, которое будет вставляться на место
плейсхолдера \%d.\par

\begin{lstlisting}[language=Kotlin, caption=\leftline{спользование ресурсов Plurals в коде}, label=lst:kot:plurals]
override fun onCreate(savedInstanceState: Bundle?) {
        super.onCreate(savedInstanceState)
        setContentView(R.layout.activity_work63)

        val seekBar = findViewById<SeekBar>(R.id.seekBar)
        val textView = findViewById<TextView>(R.id.seekBarValue)
        seekBar.setOnSeekBarChangeListener(object : SeekBar.OnSeekBarChangeListener {
            override fun onProgressChanged(seekBar: SeekBar, progress: Int, fromUser: Boolean) {
                textView.text = resources.getQuantityString(R.plurals.flowers, progress, progress)
            }
            override fun onStartTrackingTouch(seekBar: SeekBar) {}
            override fun onStopTrackingTouch(seekBar: SeekBar) {}
        })
    }
\end{lstlisting}

\section{String array}
Еще одним видом строковых ресурсов является string-array или массив строк.
Ресурс задается с помощью элемента \texttt{<string-array>}. Фактически он
определяет набор строк. А каждая отдельная строка задается с помощью
элемента \texttt{<item>}.
\begin{lstlisting}[language=xml, caption=\leftline{Объявление ресурсов string array}, label=lst:xml:array]
<string-array name="languages">
        <item>Java</item>
        <item>Kotlin</item>
        <item>Dart</item>
    </string-array>
\end{lstlisting}

С помощью метода getStringArray получаем ресурс в массив строк и затем с
помощью цикла сложили из массива одну строку и передаем ее в TextView.

\begin{lstlisting}[language=Kotlin, caption=\leftline{Использование ресурсов string array в в коде}, label=lst:kot:array]
class Work64 : AppCompatActivity() {
    override fun onCreate(savedInstanceState: Bundle?) {
        super.onCreate(savedInstanceState)
        val res = resources
        val languages = res.getStringArray(R.array.languages)
        var langs = ""
        for (lang in languages) {
            langs += "$lang "
        }
        val textView = TextView(this)
        textView.text = langs
        textView.textSize = 28f
        setContentView(textView)
    }
}
\end{lstlisting}
\img{img4}{Ресурсы string array}

\section{Ресурсы dimension}
Определение размеров должно находиться в каталоге res/values в файле с
любым произвольным именем.

\begin{lstlisting}[language=xml, caption=\leftline{Объявление ресурсов dimension}, label=lst:xml:dimension]
<resources>
    <dimen name="fab_margin">16dp</dimen>
    <dimen name="horizontal_margin">64dp</dimen>
    <dimen name="vertical_margin">32dp</dimen>
    <dimen name="text_size">32sp</dimen>
</resources>
\end{lstlisting}

Как и другие ресурсы, ресурс dimension определяется в корневом элементе
\texttt{<resources>}. Тег \texttt{<dimen>} обозначает ресурс и в качестве
значния принимает некоторое значение размера в одной из принятых единиц
измерения (dp, sp, pt, px, mm, in). Более подробно установка размеров
рассматривалась в одной из прошлых практических занятий.\par

\begin{lstlisting}[language=xml, caption=\leftline{Использование ресурсов dimension в XML-коде}, label=lst:xml:dimension:use]
<TextView
        android:id="@+id/textView"
        android:layout_width="wrap_content"
        android:layout_height="wrap_content"
        android:text="Hello MIREA"
        android:background="#eaeaea"

        android:layout_marginTop="@dimen/vertical_margin"
        android:layout_marginStart="@dimen/horizontal_margin"
        android:textSize="@dimen/text_size"

        app:layout_constraintLeft_toLeftOf="parent"
        app:layout_constraintTop_toTopOf="parent" />
\end{lstlisting}

Для получения ресурсов в коде применяется метод \texttt{getDimension()}
класса \texttt{Resources}.
\img{img5}{Ресурсы dimension}

\section{Ресурсы Color}
В приложении Android также можно определять ресурсы цветов (Color). Они
должны храниться в файле по пути res/values и также, как и ресурсы строк,
заключены в тег \texttt{<resources>}. Так, по умолчанию при создании самого
простого проекта в папку res/values добавляется файл colors.xml

\begin{lstlisting}[language=xml, caption=\leftline{Объявление ресурсов colors}, label=lst:xml:color]
<?xml version="1.0" encoding="utf-8"?>
<resources>
    <color name="purple_200">#FFBB86FC</color>
    <color name="purple_500">#FF6200EE</color>
    <color name="purple_700">#FF3700B3</color>
    <color name="teal_200">#FF03DAC5</color>
    <color name="teal_700">#FF018786</color>
    <color name="black">#FF000000</color>
    <color name="white">#FFFFFFFF</color>
</resources>
\end{lstlisting}


Цвет определяется с помощью элемента \texttt{<color>}. Его атрибут
\texttt{name} устанавливает название цвета, которое будет использоваться
в приложении, а шестнадцатеричное число --- значение цвета.
Для задания цветовых ресурсов можно использовать следующие форматы:

\begin{itemize}
	\item \#RGB (\#F00 --- 12-битное значение)
	\item \#ARGB (\#8F00 --- 12-битное значение с добавлением альфа-канала)
	\item \#RRGGBB (\#FF00FF --- 24-битное значение)
	\item \#AARRGGBB (\#80FF00FF --- 24-битное значение с добавлением
		альфа-канала)
\end{itemize}


\begin{lstlisting}[language=xml, caption=\leftline{Объявление ресурсов colors}, label=lst:xml:use]
<TextView
        android:id="@+id/textView"
        android:layout_width="wrap_content"
        android:layout_height="wrap_content"
        android:text="Hello Android"

        android:background="@color/purple_500"
        android:textColor="@color/purple_700"

        android:textSize="32sp"
        app:layout_constraintLeft_toLeftOf="parent"
        app:layout_constraintTop_toTopOf="parent" />
\end{lstlisting}

\img{img6}{Ресурсы Color}
\clearpage

\section*{\LARGE{Вывод}}
\addcontentsline{toc}{section}{Вывод}
В этой практической работе были получены знания для создания и
использования различный ресурсов. Освоены такие виды ресурсов как:

\begin{itemize}
	\item ресурсы строк;
	\item форматирование строк;
	\item ресурсы Plurals;
	\item ресурсы string array;
	\item ресурсы dimension;
	\item ресурсы Color.
\end{itemize}

