\section*{\LARGE{Цель практической работы}}
\addcontentsline{toc}{section}{Цель практической работы}
В этой практической работе рассмотрим создание первого приложения для Android.
Научимся создавать проекты и запускать приложение в режиме отладки.
Так же узнаем об основах разработки приложения для Android, включая создание простого пользовательского интерфейса и
обработку ввода информации пользователем.

\newpage
\section*{\LARGE{Выполнение практической работы}}
\addcontentsline{toc}{section}{Выполнение практической работы}

\section{Создание проекта в Android Studio}
Чтобы создать новый проект в Android Studio, надо отрыть данное приложение.
Щелкнуть на кнопку New Project на экране приветствия.
Или если открыт другой проект, щелкнуть меню File и выбрать New Project.

\img{./gall/png_1.png}{Создание проекта в Android Studio}

В предложенном списке стартовых activity выбираем наиболее подходящий и далее необходимо заполнить поля в окне
Configure your new project.

\img{./gall/png_2.png}{Настройка проекта}

\newpage

\section{Создание разметки}
Создание разметки в XML файлах предпочтительнее, чем в исходном коде по нескольким причинам, но главным образом из-за
необходимости создания различных файлов разметки для устройств с различными размерами экрана.

\subsection{Создание линейной разметки (Linear Layout)}
\begin{enumerate}
	\item В Android Studio открыли файл res/layout/activity\_my.xml
	\item В окне предпросмотра щелкнули по иконке Hide , чтобы скрыть окно.
	\item Удалили элемент <TextView>.
	\item Изменили элемент <RelativeLayout> на <LinearLayout>.
	\item Добавили атрибут android:orientation и установите для него значение \("\)horizontal\("\).
	\item Удалите атрибуты android:padding и tools:context.
\end{enumerate}

\img{./gall/png_3.png}{Настройка разметки}

\subsection{Добавление текстового поля}
Как и для каждого объекта типа View, необходимо указать некоторые XML атрибуты, характерные для элемента EditText.
\begin{enumerate}
	\item В файле activity\_my.xml, создали внутри <LinearLayout> элемент <EditText> и указали для него атрибут android:id со значением @+id/textView;
	\item Создали атрибуты layout\_width и layout\_height со значением «match\_parent» и «wrap\_content», соответственно;
	\item Создали атрибут hint и указали в качестве значения строковый объект с названием input\_suggestion.
\end{enumerate}
В результате элемент <EditText> должен выглядеть следующим образом:

\img{./gall/png_4.png}{Настройка текстового поля}

\subsection{Объекты ресурсов}
Объект ресурса имеет уникальное целочисленное значение, связанное с ресурсами приложения, такими как растровые
изображения, файлы разметки или строки.
По умолчанию строковые ресурсы хранятся в файле res/values/strings.xml.
Добавили новый ресурс \("\)input\_suggestion\("\) и указали для него значение \("\)Введите свой текст\("\).
\begin{enumerate}
	\item В Android Studio открыли файл res/values/strings.xml;
	\item Добавили строку с названием \("\)input\_suggestion\("\) и значением “Введите свой текс”;
	\item Добавили строку с названием \("\)buttonText\_view\("\) и значением “Отправить”;
	\item Удалили строку, содержащую надпись \("\)hello world\("\).
\end{enumerate}
\img{./gall/png_5.png}{Создание ресурсов}

\subsection{Добавление кнопки}
\begin{enumerate}
	\item В Android Studio открыли файл res/layout/activity\_my.xml.;
	\item Создали внутри <LinearLayout> элемент <Button> сразу после элемента <EditText>;
	\item Задали ширину как \("\)wrap\_content\("\), чтобы она зависела от надписи внутри кнопки;
	\item Добавили атрибут android:text и указали в качестве значения строковый ресурс \("\)buttonText\_view\("\).
\end{enumerate}

\img{./gall/png_6.png}{Создание кнопки}


\subsection{Растягиваем поле ввода}
Для того, чтобы растянуть элемент EditText на все свободное пространство в файле activity\_my.xml добавили элементу
<EditText> атрибут layout\_weight со значением 1, а атрибуту layout\_width установим значение 0dp.

\img{./gall/png_7.png}{Растянули поля ввода}

\subsection{Запуск другого явление}
Другое явление будет запускаться при нажатии кнопки в текущем.
Для этого в классе объявили объект кнопки и присвоили ему найденную по id кнопку расположенную в разметке.
Далее установили обработчик событий на эту кнопу.
Так что при ее нажатии, будет создаваться намеренье, которому передается текущий контекст и класс явление,
которое будет запускаться.
Также в намеренье, через метод putExtra, сохраниться текст введенный пользователем в поля ввода.
И в конце вызывается метод startActivity с ново созданным намереньем.

\img{./gall/png_8.png}{Запуск нового явления}

\subsection{Создание нового явления в Android Studio}
\begin{enumerate}
	\item В Android Studio щелкнули правой кнопкой мыши по пакету java/com.mycompany.myfirstapp и выбрали New > Activity > Blank Activity;
	\item В окне Choose options настроили новое явление и нажали Finish
	\item Открыли появившийся файл MainActivity2.java и удалили все лишние методы, оставив только onCreate.
\end{enumerate}

\subsection{Отображение сообщения в новом явлении}
Чтобы отобразить сообщение, нужно:
\begin{enumerate}
	\item Создать объект TextView;
	\item В классе создать соответствующий объект текстовой надписи;
	\item Получить объект намеренья;
	\item Установить текст виджету, получив текст из намеренья с помощью объекта intent.
\end{enumerate}
\img{./gall/png_9.png}{Код java для отображение сообщения}
\img{./gall/png_10.png}{Разметка для отображение сообщения}

\newpage

\subsection*{Вывод}
\addcontentsline{toc}{section}{Вывод}
В ходе практической работы было создано первое приложение для Android.
Научились создавать проекты и запускать приложение в режиме отладки.
Так же узнали об основах разработки приложения для Android, такие как
создание разметки с текстовыми полями, полями ввода текста и кнопки,
и обработку ввода информации пользователем.