\graphicspath{{./png}}
\section*{\LARGE{Цель практической работы}}
\addcontentsline{toc}{section}{Цель практической работы}
В практической работе было рассмотренно как использовать базовые возможности
платформы для различных способов разметки экрана.

\newpage

\section*{\LARGE{Выполнение практической работы}}
\addcontentsline{toc}{section}{Выполнение практической работы}

\section{LinearLayout}
Контейнер \texttt{LinearLayout} представляет простейший контейнер - объект
\texttt{ViewGroup}, который упорядочивает все дочерние элементы в одном
направлении: по горизонтали или по вертикали.
Все элементы расположены один за другим.
Направление разметки указывается с помощью атрибута
\texttt{android:orientation}.
Если, например, ориентация разметки вертикальная
\texttt{(android:orientation="vertical")}, то все элементы располагаются в столбик - по
одному элементу на каждой строке.
Если ориентация горизонтальная
\texttt{(android:orientation="horizontal")}, то элементы располагаются в одну строку.
Например, расположим элементы в горизонтальный ряд:

\begin{lstlisting}[language=xml, caption=\leftline{xml}, label=lst:LinearLayout]
<?xml version="1.0" encoding="utf-8"?>
<LinearLayout xmlns:android="http://schemas.android.com/apk/res/android"
android:layout_width="match_parent"
android:layout_height="match_parent"
android:orientation="horizontal" >
<TextView
android:layout_width="wrap_content"
android:layout_height="wrap_content"
android:layout_margin="5dp"
android:text="Hello"
android:textSize="26sp" />
<TextView
android:layout_width="wrap_content"
android:layout_height="wrap_content"
android:layout_margin="5dp"
android:text="Android"
android:textSize="26sp" />
<TextView
android:layout_width="wrap_content"
android:layout_height="wrap_content"
android:layout_margin="5dp"
android:text="World"
android:textSize="26sp" />
</LinearLayout>
\end{lstlisting}

\img{png_1.png}{LinearLayout}

\subsection{Вес элемента}
\texttt{LinearLayout} поддерживает такое свойство, как вес элемента, которое
передается атрибутом \texttt{android:layout\_weight}.
Это свойство принимает
значение, указывающее, какую часть оставшегося свободного места
контейнера по отношению к другим объектам займет данный элемент.
Например, если один элемент у нас будет иметь для свойства
\texttt{android:layout\_weight} значение 2, а другой - значение 1, то в сумме они дадут
3, поэтому первый элемент будет занимать 2/3 оставшегося пространства, а
второй - 1/3.
Если все элементы имеют значение \texttt{android:layout\_weight="1"}, то все эти
элементы будут равномерно распределены по всей площади контейнера:

\begin{lstlisting}[language=xml, caption=\leftline{xml}, label=lst:Вес элемента]
<?xml version="1.0" encoding="utf-8"?>
<LinearLayout xmlns:android="http://schemas.android.com/apk/res/android"
 android:layout_width="match_parent"
 android:layout_height="match_parent"
 android:orientation="vertical" >
 <TextView
 android:layout_width="match_parent"
 android:layout_height="0dp"
 android:text="Hello"
 android:background="#e0e0e0"
 android:layout_weight="1"
 android:textSize="26sp" />
 <TextView
 android:layout_width="match_parent"
 android:layout_height="0dp"
 android:background="#eeeeee"
 android:text="Android"
 android:layout_weight="1"
 android:textSize="26sp" />
 <TextView
 android:layout_width="match_parent"
 android:layout_height="0dp"
 android:text="World"
 android:background="#bdbdbd"
 android:layout_weight="1"
 android:textSize="26sp" />
</LinearLayout>
\end{lstlisting}

\img{png_2.png}{Вес элемента}

При этом так как у нас вертикальный стек, то нам надо также установить для
свойства layout\_height значение 0dp.
Если бы LinearLayout имел
горизонтальную ориентацию, то для свойства layout\_width надо было бы
установить значение 0dp.
Еще один атрибут android:weightSum позволяет указать сумму весов всех
элементов.
Например:
\begin{lstlisting}[language=xml, caption=\leftline{xml}, label=lst:android:weightSum]
<?xml version="1.0" encoding="utf-8"?>
<LinearLayout xmlns:android="http://schemas.android.com/apk/res/android"
 android:layout_width="match_parent"
 android:layout_height="match_parent"
 android:orientation="vertical"
 android:weightSum="7">
 <TextView
 android:layout_width="match_parent"
 android:layout_height="0dp"
 android:text="Hello"
 android:background="#e0e0e0"
 android:layout_weight="1"
 android:textSize="26sp" />
 <TextView
 android:layout_width="match_parent"
 android:layout_height="0dp"
 android:background="#eeeeee"
 android:text="Android"
 android:layout_weight="3"
 android:textSize="26sp" />
 <TextView
 android:layout_width="match_parent"
 android:layout_height="0dp"
 android:text="World"
 android:background="#bdbdbd"
 android:layout_weight="2"
 android:textSize="26sp" />
</LinearLayout>
\end{lstlisting}
\img{png_3.png}{weightSum}

\subsection{Программное создание LinearLayout}
Создание LinearLayout в коде:
\begin{lstlisting}[language=Kotlin, caption=\leftline{Kotlin}, label=lst:Программное создание LinearLayout]
override fun onCreate(savedInstanceState: Bundle?) {
        super.onCreate(savedInstanceState)
        var linearLayout = LinearLayout(this);
        linearLayout.orientation = LinearLayout.VERTICAL
        val textView1 = TextView(this)
        textView1.text = "Hello"
        textView1.textSize = 30f
        linearLayout.addView(
            textView1,
            LinearLayout.LayoutParams(
                LinearLayout.LayoutParams.MATCH_PARENT, 0, 3f)
        )
        val textView2 = TextView(this)
        textView2.text = "Android"
        textView2.setBackgroundColor(0xFFBDBDBD.toInt())
        textView2.textSize = 30f
        linearLayout.addView(
            textView2,
            LinearLayout.LayoutParams(
                LinearLayout.LayoutParams.MATCH_PARENT, 0, 2f)
        )
        setContentView(linearLayout)
    }
\end{lstlisting}
\img{png_4}{Программное создание LinearLayout}

\section{Layout\_gravity}
Атрибут layout\_gravity позволяет устанавливать позиционирование
относительно LinearLayout.
Он принимает следуюшие значения:
\begin{itemize}
    \item top: выравнивает элемент по верхней границе контейнера;
    \item bottom: выравнивает элемент по нижней границе контейнера;
    \item left: выравнивает элемент по левой границе контейнера;
    \item right: выравнивает элемент по правой границе контейнера;
    \item center\_vertical: выравнивает элемент по центру по вертикали;
    \item center\_horizontal: выравнивает элемент по центру по горизонтали;
    \item center: элемент позиционируется в центре;
    \item fill\_vertical: элемент растягивается по вертикали;
    \item fill\_horizontal: элемент растягивается по горизонтали;
    \item fill: элемент заполняет все пространство контейнера;
    \item clip\_vertical: обрезает верхнюю и нижнюю границу элемента;
    \item clip\_horizontal: обрезает правую и левую границу элемента;
    \item start: элемент позиционируется в начале (в верхнем левом углу) контейнера;
    \item end: элемент позиционируется в конце контейнера (в верхнем правом
    углу).
\end{itemize}
Например:
\begin{lstlisting}[language=xml, caption=\leftline{xml}, label=lst:Layoutgravity2]
<LinearLayout
 xmlns:android="http://schemas.android.com/apk/res/android"
 android:layout_width="match_parent"
 android:layout_height="match_parent"
 android:orientation="vertical">
 <TextView
 android:layout_gravity="left"
 android:layout_width="wrap_content"
 android:layout_height="wrap_content"
 android:textSize="30sp"
 android:text="Hello Java!"
 android:background="#e8eaf6"/>
 <TextView
 android:layout_gravity="center"
 android:layout_width="wrap_content"
 android:layout_height="wrap_content"
 android:textSize="30sp"
 android:text="Hello World!"
 android:background="#e8eaf6"/>
 <TextView
 android:layout_gravity="right"
 android:layout_width="wrap_content"
 android:layout_height="wrap_content"
 android:textSize="30sp"
 android:text="Hello Android!"
 android:background="#e8eaf6"/>
 <TextView
 android:layout_gravity="center"
 android:layout_width="wrap_content"
 android:layout_height="wrap_content"
 android:textSize="30sp"
 android:text="Hello Kotlin!"
 android:background="#e8eaf6"/>
</LinearLayout>
\end{lstlisting}

В данном случае первый элемент TextView будет позиционироваться по
левой стороне контейнера (android:layout\_gravity=\("\)left\("\)), второй TextView по
центру (android:layout\_gravity=\("\)center\("\)), третий - по правой стороне
(android:layout\_gravity=\("\)right\("\)) и четвертый - по центру
(android:layout\_gravity=\("\)center\("\))

\img{png_5}{Layout\_gravity}
Стоит учитывать ориентацию контейнера.
Например, при вертикальной
ориентации все элементы будут представлять вертикальный стек, идущий
сверху вниз.
Поэтому значения, которые относятся к позиционированию
элемента по вертикали (например, top или bottom) никак не будут влиять на
элемент.
Также при горизонтальной ориентации LinearLayout не окажут
никакого влияния значения, которые позиционируют элемент по
горизонтали, например, left и right.
Для установки программно параметра layout\_gravity надо задать поле
gravity у объекта LinearLayout.LayoutParams:
\begin{lstlisting}[language=Kotlin, caption=\leftline{Kotlin}, label=lst:Layoutgravity1]
package com.example.mirea_mob_4sem

import android.os.Bundle
import android.view.Gravity
import android.widget.LinearLayout
import android.widget.TextView
import androidx.appcompat.app.AppCompatActivity


class MainActivity : AppCompatActivity() {
    override fun onCreate(savedInstanceState: Bundle?) {
        super.onCreate(savedInstanceState)
        val linearLayout = LinearLayout(this)
        linearLayout.orientation = LinearLayout.VERTICAL
        val layoutParams = LinearLayout.LayoutParams(
            LinearLayout.LayoutParams.WRAP_CONTENT,
            LinearLayout.LayoutParams.WRAP_CONTENT
        )
        layoutParams.gravity = Gravity.CENTER
        val textView1 = TextView(this)
        textView1.text = "Hello"
        textView1.textSize = 30f
        linearLayout.addView(textView1, layoutParams)
        setContentView(linearLayout)
    }
}
\end{lstlisting}
В качестве значения передается одна из констант класса Gravity, которые
аналогичны значениям атрибута.

\section{RelativeLayout}
RelativeLayout представляет объект ViewGroup, который располагает
дочерние элементы относительно позиции других дочерних элементов
разметки или относительно области самой разметки RelativeLayout.
Используя относительное позиционирование, мы можем установить элемент
по правому краю или в центре или иным способом, который предоставляет
данный контейнер.
Для установки элемента в файле xml предоставляет
данный контейнер.
Для установки элемента в файле xml мы можем
применять следующие атрибуты:
\begin{itemize}
    \item android:layout\_above: располагает элемент над элементом с
    указанным Id;
    \item android:layout\_below: располагает элемент под элементом с
    указанным Id;
    \item android:layout\_toLeftOf: располагается слева от элемента с
    указанным Id;
    \item android:layout\_toRightOf: располагается справа от элемента с
    указанным Id;
    \item android:layout\_toStartOf: располагает начало текущего элемента, где
    начинается элемент с указанным Id;
    \item android:layout\_toEndOf: располагает начало текущего элемента, где
    завершается элемент с указанным Id;
    \item android:layout\_alignBottom: выравнивает элемент по нижней
    границе другого элемента с указанным Id;
    \item android:layout\_alignLeft: выравнивает элемент по левой границе
    другого элемента с указанным Id;
    \item android:layout\_alignRight: выравнивает элемент по правой границе
    другого элемента с указанным Id'
    \item android:layout\_alignStart: выравнивает элемент по линии, у которой
    начинается другой элемент с указанным Id;
    \item android:layout\_alignEnd: выравнивает элемент по линии, у которой
    завершается другой элемент с указанным Id;
    \item android:layout\_alignTop: выравнивает элемент по верхней границе
    другого элемента с указанным Id;
    \item android:layout\_alignBaseline: выравнивает базовую линию элемента
    по базовой линии другого элемента с указанным Id;
    \item android:layout\_alignParentBottom: если атрибут имеет значение
    true, то элемент прижимается к нижней границе контейнера;
    \item android:layout\_alignParentRight: если атрибут имеет значение true,
    то элемент прижимается к правому краю контейнера;
    \item android:layout\_alignParentLeft: если атрибут имеет значение true, то
    элемент прижимается к левому краю контейнера;
    \item android:layout\_alignParentStart: если атрибут имеет значение true,
    то элемент прижимается к начальному краю контейнера (при
    левосторонней ориентации текста - левый край);
    \item android:layout\_alignParentEnd: если атрибут имеет значение true, то
    элемент прижимается к конечному краю контейнера (при
    левосторонней ориентации текста - правый край);
    \item android:layout\_alignParentTop: если атрибут имеет значение true, то
    элемент прижимается к верхней границе контейнера;
    \item android:layout\_centerInParent: если атрибут имеет значение true, то
    элемент располагается по центру родительского контейнера;
    \item android:layout\_centerHorizontal: при значении true выравнивает
    элемент по центру по горизонтали;
    \item android:layout\_centerVertical: при значении true выравнивает
    элемент по центру по вертикали;
\end{itemize}
Например, позиционирование относительно контейнера RelativeLayout:
\begin{lstlisting}[language=xml, caption=\leftline{xml}, label=lst:RelativeLayout]
<?xml version="1.0" encoding="utf-8"?>
<RelativeLayout
 xmlns:android="http://schemas.android.com/apk/res/android"
 android:layout_width="match_parent"
 android:layout_height="match_parent">
 <TextView android:text="Left Top"
 android:layout_height="wrap_content"
 android:layout_width="wrap_content"
 android:textSize="26sp"
 android:layout_alignParentLeft="true"
 android:layout_alignParentTop="true" />
 <TextView android:text="Right Top"
 android:layout_height="wrap_content"
 android:layout_width="wrap_content"
 android:textSize="26sp"
 android:layout_alignParentRight="true"
 android:layout_alignParentTop="true" />
 <TextView android:text="Left Bottom"
 android:layout_height="wrap_content"
 android:layout_width="wrap_content"
 android:textSize="26sp"
 android:layout_alignParentLeft="true"
 android:layout_alignParentBottom="true" />
 <TextView android:text="Right Bottom"
 android:layout_height="wrap_content"
 android:layout_width="wrap_content"
 android:textSize="26sp"
 android:layout_alignParentRight="true"
 android:layout_alignParentBottom="true" />
</RelativeLayout>
\end{lstlisting}

\img{png_6}{RelativeLayout}
Для позиционирования относительно другого элемента, нам надо указать id
этого элемента.
Так, поместим на RelativeLayout текстовое поле и кнопку:
\begin{lstlisting}[language=xml, caption=\leftline{xml}, label=lst:RelativeLayout1]
<?xml version="1.0" encoding="utf-8"?>
<RelativeLayout
 xmlns:android="http://schemas.android.com/apk/res/android"
 android:layout_width="match_parent"
 android:layout_height="match_parent">
 <EditText
 android:id="@+id/edit_message"
 android:layout_width="match_parent"
 android:layout_height="wrap_content"
 android:layout_centerInParent="true"/>
 <Button
 android:layout_width="wrap_content"
 android:layout_height="wrap_content"
 android:text="Отправить"
 android:layout_alignRight="@id/edit_message"
 android:layout_below="@id/edit_message"
 />
</RelativeLayout>
\end{lstlisting}
В данном случае поле EditText располагается по центру в RelativeLayout, а
кнопка помещается под EditText и выравнивается по его правой границе:
\img{png_7}{Позиционирование RelativeLayout в Android}
\subsection{Программное создание RelativeLayout}
Создадим элемент RelativeLayout программно в коде:
\begin{lstlisting}[language=Kotlin, caption=\leftline{Kotlin}, label=lst:RelativeLayout2]
override fun onCreate(savedInstanceState: Bundle?) {
        super.onCreate(savedInstanceState)
        val relativeLayout = RelativeLayout(this)
        val editText = EditText(this)
        editText.id = EditText.generateViewId()
        val button = Button(this)
        button.setText("Отправить")
        val editTextParams = RelativeLayout.LayoutParams(
            RelativeLayout.LayoutParams.MATCH_PARENT,
            RelativeLayout.LayoutParams.WRAP_CONTENT
        )
        editTextParams.addRule(RelativeLayout.CENTER_IN_PARENT)
        relativeLayout.addView(editText, editTextParams)
        val buttonParams = RelativeLayout.LayoutParams(
            RelativeLayout.LayoutParams.WRAP_CONTENT,
            RelativeLayout.LayoutParams.WRAP_CONTENT
        )
        buttonParams.addRule(RelativeLayout.BELOW, editText.id)
        buttonParams.addRule(RelativeLayout.ALIGN_RIGHT, editText.id)
        relativeLayout.addView(button, buttonParams)
        setContentView(relativeLayout)
    }
\end{lstlisting}
\img{png_8}{Позиционирование RelativeLayout в Android}
Чтобы задать положение элемента в контейнере, применяется класс
RelativeLayout.LayoutParams.
Через конструктор устанавливаются значения
для для ширины и высоты.
Например, у элемента EditText для ширины
устанавливается значение MATCH\_PARENT, а для высоты -
WRAP\_CONTENT.
С помощью метода addRule() мы можем добавлять дополнительные правила
для позиционирования элемента.
Этот метод в качестве параметра принимает
числовую константу, которая представляет параметр позиционирования и
которая аналогична атрибуту.
Например, атрибуту
android:layout\_centerInParent будет соответствовать константа
CENTER\_IN\_PARENT, а атрибуту android:layout\_alignRight константа
ALIGN\_RIGHT.
Стоит отметить, что в целях упрощения кода для установки id у EditText
вызывается метод generateViewId();, который позволяет программно
сгенерировать id для элемента управления.
Затем установленный id передается в качестве второго параметра в метод
addRule при установке правил для кнопки:
buttonParams.addRule(RelativeLayout.BELOW, editText.getId());
Тем самым мы указываем относительно какого элемента надо задать
расположение.
\section{TableLayout}
Контейнер TableLayout структурирует элементы управления в виде таблицы
по столбцам и строкам.
Определим в файле activity\_main.xml элемент
TableLayout, который будет включать две строки и два столбца:
\begin{lstlisting}[language=xml, caption=\leftline{xml}, label=lst:TableLayout]
<TableLayout xmlns:android="http://schemas.android.com/apk/res/android"
    android:layout_width="match_parent"
    android:layout_height="match_parent">
    <TableRow>
        <TextView
            android:layout_weight="0.5"
            android:text="Логин"
            android:layout_width="wrap_content"
            android:layout_height="wrap_content" />
        <EditText
            android:layout_weight="1"
            android:layout_width="match_parent"
            android:layout_height="wrap_content" />
    </TableRow>
    <TableRow>
        <TextView
            android:layout_weight="0.5"
            android:text="Email"
            android:layout_width="wrap_content"
            android:layout_height="wrap_content" />
        <EditText
            android:layout_weight="1"
            android:layout_width="wrap_content"
            android:layout_height="wrap_content" />
    </TableRow>
</TableLayout>
\end{lstlisting}
\img{png_9}{Разметка TableLayout в Android}
Используя элемент TableRow, мы создаем отдельную строку. Как разметка
узнает сколько столбцов надо создать? Android находит строку с
максимальным количеством виджетов одного уровня, и это количество будет
означать количество столбцов. Например, в данном случае у нас определены
две строки и в каждой по два элемента. Если бы в какой-нибудь из них было
бы три виджета, то соответственно столбцов было бы также три, даже если в
другой строке осталось бы два виджета.
Причем элемент TableRow наследуется от класса LinearLayout, поэтому мы
можем к нему применять тот же функционал, что и к LinearLayout. В
частности, для определения пространства для элементов в строке
используется атрибут android:layout\_weight.
Если какой-то элемент должен быть растянут на ряд столбцов, то мы можем
растянуть его с помощью атрибута layout\_span, который указывает на какое
количество столбцов надо растянуть элемент:
\begin{lstlisting}[language=xml, caption=\leftline{xml}, label=lst:TableRow]
<TableLayout xmlns:android="http://schemas.android.com/apk/res/android"
android:layout_width="match_parent"
android:layout_height="match_parent">
<TableRow>
<TextView
android:layout_weight="0.5"
android:text="Логин"
android:layout_width="wrap_content"
android:layout_height="wrap_content" />
<EditText
android:layout_weight="1"
android:layout_width="match_parent"
android:layout_height="wrap_content" />
</TableRow>
<TableRow>
<TextView
android:layout_weight="0.5"
android:text="Email"
android:layout_width="wrap_content"
android:layout_height="wrap_content" />
<EditText
android:layout_weight="1"
android:layout_width="wrap_content"
android:layout_height="wrap_content" />
</TableRow>
</TableLayout>
\end{lstlisting}
\img{png10]}{Растягиваем элемент на несколько столбцов в TableLayout в Android}
Также можно растянуть элемент на всю строку, установив у него атрибут
android:layout\_weight=\("1"\):
\begin{lstlisting}[language=xml, caption=\leftline{xml}, label=lst:layout_weight]
<TableRow>
<Button
    android:text="Отправить"
    android:layout_width="match_parent"
    android:layout_height="wrap_content"
    android:layout_weight="1" />
</TableRow>
\end{lstlisting}
\subsection{Программное создание TableLayout}
Создадим TableLayout программным образом, переложив на код java самый
первый пример из данного материала:
\begin{lstlisting}[language=Kotlin, caption=\leftline{Kotlin}, label=lst:Программное создание TableLayout]
override fun onCreate(savedInstanceState: Bundle?) {
        super.onCreate(savedInstanceState)
        val tableLayout = TableLayout(this)

        val tableRow1 = TableRow(this)
        val textView1 = TextView(this)
        textView1.text = "Логин"
        tableRow1.addView(
            textView1, TableRow.LayoutParams(
                TableRow.LayoutParams.WRAP_CONTENT,
                TableRow.LayoutParams.WRAP_CONTENT, 0.5f
            )
        )
        val editText1 = EditText(this)
        tableRow1.addView(
            editText1, TableRow.LayoutParams(
                TableRow.LayoutParams.WRAP_CONTENT,
                TableRow.LayoutParams.WRAP_CONTENT, 1.0f
            )
        )

        val tableRow2 = TableRow(this)
        val textView2 = TextView(this)
        textView2.text = "Email"
        tableRow2.addView(
            textView2, TableRow.LayoutParams(
                TableRow.LayoutParams.WRAP_CONTENT,
                TableRow.LayoutParams.WRAP_CONTENT, 0.5f
            )
        )
        val editText2 = EditText(this)
        tableRow2.addView(
            editText2, TableRow.LayoutParams(
                TableRow.LayoutParams.WRAP_CONTENT,
                TableRow.LayoutParams.WRAP_CONTENT, 1f
            )
        )
        tableLayout.addView(tableRow1)
        tableLayout.addView(tableRow2)
        setContentView(tableLayout)
    }
\end{lstlisting}
\img{png_11}{Программное создание TableLayout}

\section{FrameLayout}
Контейнер FrameLayout предназначен для вывода на экран одного
помещенного в него визуального элемента. Если же мы поместим несколько
элементов, то они будут накладываться друг на друга. Тем не менее также
можно располагать в FrameLayout несколько элементов.
Допустим, вложим в FrameLayout два элемента TextView:
\begin{lstlisting}[language=xml, caption=\leftline{xml}, label=lst:FrameLayout]
<?xml version="1.0" encoding="utf-8"?>
<FrameLayout
 xmlns:android="http://schemas.android.com/apk/res/android"
 android:layout_width="match_parent"
 android:layout_height="match_parent">
 <TextView
 android:layout_width="wrap_content"
 android:layout_height="wrap_content"
 android:text="Hello World!"
 android:textSize="26sp"/>
 <TextView
 android:layout_width="wrap_content"
 android:layout_height="wrap_content"
 android:text="Hello Android!"
 android:textSize="26sp"
 android:layout_marginTop="50dp"/>
</FrameLayout>
\end{lstlisting}
Здесь оба элемента позиционируются в одно и то же место - в левый верхний
угол контейнера FrameLayout, и чтобы избежать наложения, в данном случае
у второго TextView устанавливается отступ сверху в 50 единиц.
\img{png_12}{FrameLayout в Android}
Нередко FrameLayout применяется для создания производных контейнеров,
например, ScrollView, который обеспечивает прокрутку.
Элементы управления, которые помещаются в FrameLayout, могут
установить свое позиционирование с помощью атрибута
android:layout\_gravity:
\begin{lstlisting}[language=xml, caption=\leftline{xml}, label=lst:layoutgravity]
<?xml version="1.0" encoding="utf-8"?>
<FrameLayout
 xmlns:android="http://schemas.android.com/apk/res/android"
 android:layout_width="match_parent"
 android:layout_height="match_parent">
 <TextView
 android:layout_width="wrap_content"
 android:layout_height="wrap_content"
 android:text="Hello World!"
 android:textSize="26sp"
 android:layout_gravity="center_horizontal" />
 <TextView
 android:layout_width="wrap_content"
 android:layout_height="wrap_content"
 android:text="Welcome to Java World"
 android:textSize="26sp"
 android:layout_gravity="center"/>
 <TextView
 android:layout_width="wrap_content"
 android:layout_height="wrap_content"
 android:text="Hello Android!"
 android:textSize="26sp"
 android:layout_gravity="bottom|center_horizontal"/>
</FrameLayout>
\end{lstlisting}
При указании значения мы можем комбинировать ряд значений, разделяя их
вертикальной чертой: bottom | center\_horizontal
\img{png_13}{Gravity в FrameLayout в Android}
Программное создание FrameLayout в коде MainActivity:

\begin{lstlisting}[language=Kotlin, caption=\leftline{Kotlin}, label=lst:Программное создание FrameLayout]
override fun onCreate(savedInstanceState: Bundle?) {
        super.onCreate(savedInstanceState)
        val frameLayout = FrameLayout(this)
        val textView = TextView(this)
        textView.text = "Hello World!"
        val layoutParams = FrameLayout.LayoutParams(
            FrameLayout.LayoutParams.WRAP_CONTENT,
            FrameLayout.LayoutParams.WRAP_CONTENT
        )
        layoutParams.gravity = Gravity.CENTER_HORIZONTAL or Gravity.TOP
        textView.layoutParams = layoutParams
        textView.textSize = 26f
        frameLayout.addView(textView)
        setContentView(frameLayout)
    }
\end{lstlisting}

\section{GridLayout}
GridLayout представляет еще один контейнер, который позволяет создавать
табличные представления. GridLayout состоит из коллекции строк, каждая из
которых состоит из отдельных ячеек:

\begin{lstlisting}[language=xml, caption=\leftline{xml}, label=lst:GridLayout]
<?xml version="1.0" encoding="utf-8"?>
<GridLayout xmlns:android="http://schemas.android.com/apk/res/android"
 android:layout_width="match_parent"
 android:layout_height="match_parent"
 android:rowCount="3"
 android:columnCount="3">
 <Button android:text="1" />
 <Button android:text="2" />
 <Button android:text="3" />
 <Button android:text="4" />
 <Button android:text="5" />
 <Button android:text="6" />
 <Button android:text="7" />
 <Button android:text="8" />
 <Button android:text="9" />
</GridLayout>
\end{lstlisting}
С помощью атрибутов android:rowCount и android:columnCount
устанавливается число строк и столбцов соответственно. Так, в данном
случае устанавливаем 3 строки и 3 столбца. GridLayout автоматически может
позиционировать вложенные элементы управления по строкам. Так, в нашем
случае первая кнопка попадает в первую ячейку (первая строка первый
столбец), вторая кнопка - во вторую ячейку и так далее.
При этом ширина столбцов устанавливается автоматически по ширине
самого широкого элемента.
\img{png_14}{GridLayout в Android}
Однако мы можем явно задать номер столбца и строки для определенного
элемента, а при необходимости растянуть на несколько столбцов или строк.
Для этого мы можем применять следующие атрибуты:
\begin{itemize}
    \item android:layout\_column: номер столбца (отсчет идет от нуля);
    \item android:layout\_row: номер строки;
    \item android:layout\_columnSpan: количество столбцов, на которые
    растягивается элемент;
    \item android:layout\_rowSpan: количество строк, на которые растягивается
    элемент.
\end{itemize}
Например:
\begin{lstlisting}[language=xml, caption=\leftline{xml}, label=lst:GridLayoutAtr]
    <GridLayout xmlns:android="http://schemas.android.com/apk/res/android"
 android:layout_width="match_parent"
 android:layout_height="match_parent"
 android:rowCount="3"
 android:columnCount="3">
 <Button
 android:text="1"
 android:layout_column="0"
 android:layout_row="0" />
 <Button android:text="2"
 android:layout_column="1"
 android:layout_row="0"/>
 <Button android:text="3"
 android:layout_column="2"
 android:layout_row="0" />
 <Button android:text="4"
 android:layout_width="180dp"
 android:layout_columnSpan="2"/>
 <Button android:text="5"
 android:layout_height="100dp"
 android:layout_rowSpan="2"/>
 <Button android:text="6" />
 <Button android:text="7"/>
</GridLayout>
\end{lstlisting}
\img{png_15}{Растяжение строк и столбцов в GridLayout в Android}
\subsection{Программное создание GridLayout}
Среди методов GridLayout следует отметить методы setRowCount() и
setColumnCount(), которые позволяют задать соответственно количество
строк и столбцов. Например, определим в коде GridLayout, аналогичнй
первому примеру в данном материале:
\begin{lstlisting}[language=Kotlin, caption=\leftline{Kotlin}, label=lst:Программное создание GridLayout]
override fun onCreate(savedInstanceState: Bundle?) {
        super.onCreate(savedInstanceState)
        val gridLayout = GridLayout(this)

        gridLayout.setRowCount(3)

        gridLayout.setColumnCount(3)
        for (i in 1..9) {
            val btn = Button(this)
            btn.setText(i.toString())
            gridLayout.addView(btn)
        }
        setContentView(gridLayout)
    }
\end{lstlisting}
В данном случае GridLayout имеет три строки и три столбца. При добавлении
виджетов (в данном случае кнопок) они последовательно помещаются в
ячейки грида по одному виджету в ячейке.
Для более детальной настройки расположения виджета в гриде можно
использовать класс GridLayout.LayoutParams. Этот класс имеет ряд
свойств, которые позволяют настроить расположение:
\begin{itemize}
    \item columnSpec: задает столбец для расположения в виде объекта
    GridLayout.Spec;
    \item rowSpec: задает строку для расположения в виде объекта
    GridLayout.Spec;
    \item leftMargin: задает отступ слева;
    \item rightMargin: задает отступ справа;
    \item topMargin: задает отступ сверху;
    \item bottomMargin: задает отступ снизу;
    \item width: задает ширину виджета;
    \item height: задает высоту виджета.
\end{itemize}
Объект GridLayout.Spec позволяет задать размещение в ячейках столбца или
строки. Для создание этого объекта применяется статический метод
GridLayout.spec(), который имеет ряд версий. Отметим среди них следующие:
\begin{itemize}
    \item GridLayout.spec(int): задает столбец или строку, где располагается
    виджет. Отсчет ячеек начинается с нуля. Виджет занимает только
    одну ячейку;
    \item GridLayout.spec(int, int): первый параметр задает столбец или
    строку, где располагается виджет. Второй параметр указывает,
    насколько ячеек растягивается виджет;
    \item GridLayout.spec(int, android.widget.GridLayout.Alignment): первый
    параметр задает столбец или строку, где располагается виджет.
    Второй параметр устанавливает выравнивание виджета;
    \item GridLayout.spec(int, int, android.widget.GridLayout.Alignment):
    первый параметр задает столбец или строку, где располагается
    виджет. Второй параметр указывает, насколько ячеек растягивается
    виджет. Третий параметр устанавливает выравнивание виджета.
\end{itemize}
Пример применения GridLayout.LayoutParams:
\begin{lstlisting}[language=Kotlin, caption=\leftline{Kotlin}, label=lst:GridLayout.LayoutParams]
val btn = Button(this)
        btn.text = "нажми"
        val layoutParams = GridLayout.LayoutParams()
        layoutParams.columnSpec = GridLayout.spec(0, 2)
        layoutParams.rowSpec = GridLayout.spec(1, 1)
        layoutParams.leftMargin = 5
        layoutParams.rightMargin = 5
        layoutParams.topMargin = 4
        layoutParams.bottomMargin = 4
        layoutParams.width = GridLayout.LayoutParams.MATCH_PARENT
        layoutParams.height = GridLayout.LayoutParams.WRAP_CONTENT
        gridLayout.addView(btn, layoutParams)
\end{lstlisting}
Например, реализуем в коде второй пример из данного материала:
\begin{lstlisting}[language=Kotlin, caption=\leftline{Kotlin}, label=lst:GridLayoutP]
override fun onCreate(savedInstanceState: Bundle?) {
        super.onCreate(savedInstanceState)
        val gridLayout = GridLayout(this)
        // количество строк
        // количество строк
        gridLayout.rowCount = 3
        // количество столбцов
        // количество столбцов
        gridLayout.columnCount = 3
        for (i in 1..3) {
            val btn = Button(this)
            btn.text = i.toString()
            gridLayout.addView(btn)
        }
        val btn4 = Button(this)
        btn4.text = "4"
        val layoutParams4 = GridLayout.LayoutParams()
        layoutParams4.columnSpec = GridLayout.spec(0, 2)
        layoutParams4.width = TypedValue.applyDimension(
            TypedValue.COMPLEX_UNIT_DIP, 180f,
            resources.displayMetrics
        ).toInt()
        gridLayout.addView(btn4, layoutParams4)
        val btn5 = Button(this)
        btn5.text = "5"
        val layoutParams5 = GridLayout.LayoutParams()
        layoutParams5.rowSpec = GridLayout.spec(1, 2)
        layoutParams5.height = TypedValue.applyDimension(
            TypedValue.COMPLEX_UNIT_DIP, 100f,
            resources.displayMetrics
        ).toInt()
        gridLayout.addView(btn5, layoutParams5)
        val btn6 = Button(this)
        btn6.text = "6"
        val btn7 = Button(this)
        btn7.text = "7"
        gridLayout.addView(btn6)
        gridLayout.addView(btn7)

        setContentView(gridLayout)
    }
\end{lstlisting}
\section{ScrollView}
Контейнер ScrollView предназначен для создания прокрутки для такого
интерфейса, все элементы которого одномоментно не могут поместиться на
экране устройства. ScrollView может вмещать только один элемент, поэтому
если мы хотим разместить несколько элементов, то их надо поместить в
какой-нибудь контейнер.
Например, определим ряд TextView с большими текстами:
\begin{lstlisting}[language=xml, caption=\leftline{xml}, label=lst:ScrollView]
<?xml version="1.0" encoding="utf-8"?>
<ScrollView
 xmlns:android="http://schemas.android.com/apk/res/android"
 android:layout_width="match_parent"
 android:layout_height="match_parent">
 <LinearLayout
 android:layout_width="match_parent"
 android:layout_height="match_parent"
 android:orientation="vertical"
 >
 <TextView
 android:layout_width="wrap_content"
 android:layout_height="wrap_content"
 android:text="What is Lorem Ipsum?"
 android:textSize="34sp" />
 <TextView
 android:layout_width="wrap_content"
 android:layout_height="wrap_content"
 android:text="Lorem Ipsum is simply dummy text of the printing and
typesetting industry...like Aldus PageMaker including versions of Lorem Ipsum."
 android:textSize="14sp"/>
 <TextView
 android:layout_width="wrap_content"
 android:layout_height="wrap_content"
 android:text="Why do we use it?"
 android:layout_marginTop="16dp"
 android:textSize="34sp"/>
 <TextView
 android:layout_width="wrap_content"
 android:layout_height="wrap_content"
 android:text="Lorem Ipsum is simply dummy text of the printing and
typesetting industry...like Aldus PageMaker including versions of Lorem Ipsum."
 android:textSize="14sp"/>
 <TextView
 android:layout_width="wrap_content"
 android:layout_height="wrap_content"
 android:text="Where can I get some?"
 android:layout_marginTop="16dp"
 android:textSize="34sp"/>
 <TextView
 android:layout_width="wrap_content"
 android:layout_height="wrap_content"
 android:text="There are many variations of passages of Lorem Ipsum
available ... or non-characteristic words etc."
 android:textSize="14sp"/>
 </LinearLayout>
</ScrollView>
\end{lstlisting}
Так как в ScrollView можно поместить только один элемент, то все TextView
заключены в LinearLayout. И если площадь экрана будет недостаточной,
чтобы поместить все содержимое LinearLayout, то станет доступной
прокрутка:
\img{png_16}{ScrollView в Android Studio}
\begin{lstlisting}[language=Kotlin, caption=\leftline{Kotlin}, label=lst:ScrollViewK]
override fun onCreate(savedInstanceState: Bundle?) {
        super.onCreate(savedInstanceState)
        val gridLayout = GridLayout(this)
        // количество строк
        // количество строк
        gridLayout.rowCount = 3
        // количество столбцов
        // количество столбцов
        gridLayout.columnCount = 3
        for (i in 1..3) {
            val btn = Button(this)
            btn.text = i.toString()
            gridLayout.addView(btn)
        }
        val btn4 = Button(this)
        btn4.text = "4"
        val layoutParams4 = GridLayout.LayoutParams()
        layoutParams4.columnSpec = GridLayout.spec(0, 2)
        layoutParams4.width = TypedValue.applyDimension(
            TypedValue.COMPLEX_UNIT_DIP, 180f,
            resources.displayMetrics
        ).toInt()
        gridLayout.addView(btn4, layoutParams4)
        val btn5 = Button(this)
        btn5.text = "5"
        val layoutParams5 = GridLayout.LayoutParams()
        layoutParams5.rowSpec = GridLayout.spec(1, 2)
        layoutParams5.height = TypedValue.applyDimension(
            TypedValue.COMPLEX_UNIT_DIP, 100f,
            resources.displayMetrics
        ).toInt()
        gridLayout.addView(btn5, layoutParams5)
        val btn6 = Button(this)
        btn6.text = "6"
        val btn7 = Button(this)
        btn7.text = "7"
        gridLayout.addView(btn6)
        gridLayout.addView(btn7)

        setContentView(gridLayout)
    }
\end{lstlisting}
\section{Вложенные layout}
Одна layout может содержать другую layout. Для этого применяется элемент
include.
Например, добавим в папку res/layout два файла layout, которые пусть будут
называться text\_panel.xml и button\_panel.xml:
\img{png_17}{Include layouts in Android}
В файле text\_panel.xml определим следующий код:
\begin{lstlisting}[language=xml, caption=\leftline{xml}, label=lst:layout]
<?xml version="1.0" encoding="utf-8"?>
<androidx.constraintlayout.widget.ConstraintLayout
 xmlns:android="http://schemas.android.com/apk/res/android"
 xmlns:app="http://schemas.android.com/apk/res-auto"
 android:layout_width="wrap_content"
 android:layout_height="wrap_content">
 <TextView
 android:id="@+id/clicksText"
 android:layout_width="wrap_content"
 android:layout_height="wrap_content"
 android:textSize="30sp"
 android:text="0 Clicks"
 app:layout_constraintLeft_toLeftOf="parent"
 app:layout_constraintTop_toTopOf="parent"
 />
</androidx.constraintlayout.widget.ConstraintLayout>
\end{lstlisting}
По сути, здесь просто определено поле TextView для вывода текста.
В файле button\_panel.xml определим следующую разметку:
\begin{lstlisting}[language=xml, caption=\leftline{xml}, label=lst:button_panel.xml]
<?xml version="1.0" encoding="utf-8"?>
<androidx.constraintlayout.widget.ConstraintLayout
 xmlns:android="http://schemas.android.com/apk/res/android"
 xmlns:app="http://schemas.android.com/apk/res-auto"
 android:layout_width="wrap_content"
 android:layout_height="wrap_content">
 <Button
 android:layout_width="wrap_content"
 android:layout_height="wrap_content"
 android:text="Click"
 android:onClick="onClick"
 app:layout_constraintLeft_toLeftOf="parent"
 app:layout_constraintTop_toTopOf="parent"
 />
</androidx.constraintlayout.widget.ConstraintLayout>
\end{lstlisting}
Здесь определена кнопка, нажатия которой мы будем обрабатывать.
Основным файлом разметки, который определяет интерфейс приложения,
по-прежнему является activity\_main.xml. Изменим его:
\begin{lstlisting}[language=xml, caption=\leftline{xml}, label=lst:text_panel.xml]
<?xml version="1.0" encoding="utf-8"?>
<androidx.constraintlayout.widget.ConstraintLayout
 xmlns:android="http://schemas.android.com/apk/res/android"
 xmlns:app="http://schemas.android.com/apk/res-auto"
 xmlns:tools="http://schemas.android.com/tools"
 android:layout_width="match_parent"
 android:layout_height="match_parent"
 android:padding="16dp"
 tools:context=".MainActivity">
 <include
 android:id="@+id/textView"
 layout="@layout/text_panel"
 android:layout_width="wrap_content"
 android:layout_height="wrap_content"
 app:layout_constraintLeft_toLeftOf="parent"
 app:layout_constraintTop_toTopOf="parent"
 app:layout_constraintBottom_toTopOf="@+id/button"
 />
 <include
 android:id="@+id/button"
 layout="@layout/button_panel"
 android:layout_width="wrap_content"
 android:layout_height="wrap_content"
 app:layout_constraintLeft_toLeftOf="parent"
 app:layout_constraintTop_toBottomOf="@+id/textView"
 />
</androidx.constraintlayout.widget.ConstraintLayout>
\end{lstlisting}
С помощью ConstraintLayout весь интерфейс здесь организуется в виде
вертикального стека. С помощью элементов include внутрь ConstraintLayout
добавляется содержимое файлов text\_panel.xml и button\_panel.xml. Для
указания названия файла применяется атрибут layout.
Это все равно, что, если бы мы напрямую вместо элемента include добавили
содержимое файлов. Однако такой способ имеет свои преимущества.
Например, какая-то часть разметки, группа элементов управления может
повторяться в различных activity. И чтобы не определять по сто раз эти
элементы, можно вынести их в отдельный файл layout и с помощью include
подключать их.
После добавления в ConstraintLayout к элементам include можно применять
все те стандартные атрибуты, которые применяются в этом контейнере к
вложенным элементам, например, настроить размеры, расположение. Также
стоит отметить, что добавлять внешние layout можно не только в
ConstraintLayout, но и в другие контейнеры (LinearLayout, RelativeLayout и
т.д.)
\begin{lstlisting}[language=Kotlin, caption=\leftline{Kotlin}, label=lst:clicks]
package com.example.mirea_mob_4sem

import android.os.Bundle
import android.view.View
import android.widget.Button
import android.widget.EditText
import android.widget.RelativeLayout
import android.widget.TextView
import androidx.appcompat.app.AppCompatActivity


class MainActivity : AppCompatActivity() {
private var clicks : Int = 0
override fun onCreate(savedInstanceState: Bundle?) {
super.onCreate(savedInstanceState)
setContentView(R.layout.activity4)
}

fun onClick(view: View?) {
val clicksText = findViewById<TextView>(R.id.clicksText)
clicks++
clicksText.text = "$clicks Clicks"
}

}
\end{lstlisting}
В MainActivity мы можем обращаться к элементам во вложенных файлах
layout. Например, мы можем установить обработчик нажатия кнопки, в
котором при нажатии изменять текст в TextView.
\img{png_18}{Вложенные файлы layout в Activity в Android}
При этом мы несколько раз можем добавлять в один файл layout другой файл
layout. Для этого вначале изменим файл button\_panel.xml следующим
образом:
\begin{lstlisting}[language=xml, caption=\leftline{xml}, label=lst:button_panel2.xml]
<?xml version="1.0" encoding="utf-8"?>
<androidx.constraintlayout.widget.ConstraintLayout
 xmlns:android="http://schemas.android.com/apk/res/android"
 xmlns:app="http://schemas.android.com/apk/res-auto"
 android:layout_width="wrap_content"
 android:layout_height="wrap_content"
 android:background="#3F51B5"
 android:paddingTop="10dp"
 android:paddingBottom="10dp">
 <Button
 android:id="@+id/clickBtn"
 android:layout_width="wrap_content"
 android:layout_height="wrap_content"
 app:layout_constraintLeft_toLeftOf="parent"
 app:layout_constraintTop_toTopOf="parent"
 />
</androidx.constraintlayout.widget.ConstraintLayout>
\end{lstlisting}
И изменим файл activity\_main.xml:
\begin{lstlisting}[language=xml, caption=\leftline{xml}, label=lst:activity\_main.xml]
<?xml version="1.0" encoding="utf-8"?>
<androidx.constraintlayout.widget.ConstraintLayout
 xmlns:android="http://schemas.android.com/apk/res/android"
 xmlns:app="http://schemas.android.com/apk/res-auto"
 xmlns:tools="http://schemas.android.com/tools"
 android:layout_width="match_parent"
 android:layout_height="match_parent"
 android:padding="16dp"
 tools:context=".MainActivity">
 <include
 layout="@layout/text_panel"
 android:layout_width="wrap_content"
 android:layout_height="wrap_content"
 app:layout_constraintLeft_toLeftOf="parent"
 app:layout_constraintTop_toTopOf="parent"
 />
 <include layout="@layout/button_panel"
 android:id="@+id/plus_button"
 android:layout_width="wrap_content"
 android:layout_height="wrap_content"
 app:layout_constraintLeft_toLeftOf="parent"
 app:layout_constraintBottom_toBottomOf="parent"
 app:layout_constraintRight_toLeftOf="@+id/minus_button"/>
 <include layout="@layout/button_panel"
 android:id="@+id/minus_button"
 android:layout_width="wrap_content"
 android:layout_height="wrap_content"
 android:layout_marginLeft="36dp"
 app:layout_constraintLeft_toRightOf="@id/plus_button"
 app:layout_constraintBottom_toBottomOf="parent"/>
</androidx.constraintlayout.widget.ConstraintLayout>
\end{lstlisting}
Теперь файл button_panel.xml добавляется два раза. Важно, что при
добавлении этого файла каждому элементу include присвоен определенный
id. По этому id мы сможем узнать, о каком именно элементе include идет
речь.
\img{png_19}{вложение layout с помощью include в android}

\section{Gravity}
Атрибут gravity задает позиционирование содержимого внутри визуального
элемента. Он может принимать следующие значения
\begin{itemize}
    \item top: элементы размещаются вверху;
    \item bottom: элементы размещаются внизу;
    \item left: элементы размещаются в левой стороне;
    \item right: элементы размещаются в правой стороне контейнера;
    \item center\_vertical: выравнивает элементы по центру по вертикали;
    \item center\_horizontal: выравнивает элементы по центру по горизонтали;
    \item center: элементы размещаются по центру;
    \item fill\_vertical: элемент растягивается по вертикали;
    \item fill\_horizontal: элемент растягивается по горизонтали;
    \item fill: элемент заполняет все пространство контейнера;
    \item clip\_vertical: обрезает верхнюю и нижнюю границу элементов;
    \item clip\_horizontal: обрезает правую и левую границу элементов;
    \item start: элемент позиционируется в начале (в верхнем левом углу);
    контейнера;
    \item end: элемент позиционируется в конце контейнера(в верхнем правом
    углу).
\end{itemize}
Например, поместим текст в самый низ в элементе TextView:
\begin{lstlisting}[language=xml, caption=\leftline{xml}, label=lst:Gravity]
<?xml version="1.0" encoding="utf-8"?>
<androidx.constraintlayout.widget.ConstraintLayout
 xmlns:android="http://schemas.android.com/apk/res/android"
 xmlns:app="http://schemas.android.com/apk/res-auto"
 xmlns:tools="http://schemas.android.com/tools"
 android:layout_width="match_parent"
 android:layout_height="match_parent"
 tools:context=".MainActivity">
 <TextView
 android:gravity="bottom"

 android:layout_width="0dp"
 android:layout_height="200dp"
 android:text="Hello Android!"
 android:textSize="30sp"
 android:background="#e8eaf6"
 app:layout_constraintLeft_toLeftOf="parent"
 app:layout_constraintRight_toRightOf="parent"
 app:layout_constraintTop_toTopOf="parent" />
</androidx.constraintlayout.widget.ConstraintLayout>
\end{lstlisting}
\img{png_20}{Gravity в Android Studio}
При необходимости мы можем комбинировать значения, разделяя их
вертикальной чертой:
\begin{lstlisting}[language=xml, caption=\leftline{xml}, label=lst:Gravity2]
<TextView
 android:gravity="bottom | right"

 android:layout_width="0dp"
 android:layout_height="200dp"
 android:text="Hello Android!"
 android:textSize="30sp"
 android:background="#e8eaf6"
 app:layout_constraintLeft_toLeftOf="parent"
 app:layout_constraintRight_toRightOf="parent"
 app:layout_constraintTop_toTopOf="parent" />
\end{lstlisting}
\img{png_21}{Gravity right bottom в Android}
\subsection{Программная установка gravity}
Чтобы установить параметр gravity у элемента надо вызвать метод
setGravity(). В качестве параметра в метод передается одна из констант
класса Gravity, которые аналогичны значениям атрибута (за тем
исключенем, что названия в верхнем регистре):
\begin{lstlisting}[language=Kotlin, caption=\leftline{Kotlin}, label=lst:gravity]
    class MainActivity : AppCompatActivity() {
    private var clicks : Int = 0
    override fun onCreate(savedInstanceState: Bundle?) {
        super.onCreate(savedInstanceState)
        val constraintLayout = ConstraintLayout(this)
        val textView = TextView(this)
        textView.text = "Hello Android!"
        textView.textSize = 30f
        textView.setBackgroundColor(-0x17150a)
        textView.gravity = Gravity.CENTER
        val layoutParams =
            ConstraintLayout.LayoutParams(ConstraintLayout.LayoutParams.MATCH_CONSTRAINT, 200)
        layoutParams.leftToLeft = ConstraintLayout.LayoutParams.PARENT_ID
        layoutParams.rightToRight = ConstraintLayout.LayoutParams.PARENT_ID
        layoutParams.topToTop = ConstraintLayout.LayoutParams.PARENT_ID
        layoutParams.bottomToBottom = ConstraintLayout.LayoutParams.PARENT_ID
        textView.layoutParams = layoutParams
        constraintLayout.addView(textView)
        setContentView(constraintLayout)
    }

}
\end{lstlisting}
\img{png_22}{Программная установка layout\_gravity в Android}