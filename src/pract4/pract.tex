\graphicspath{{./png}}
\section*{\LARGE{Цель практической работы}}
\addcontentsline{toc}{section}{Цель практической работы}
В практической работе было рассмотренно как использовать базовые возможности
платформы для различных способов разметки экрана.

\newpage

\section*{\LARGE{Выполнение практической работы}}
\addcontentsline{toc}{section}{Выполнение практической работы}

\section{LinearLayout}
Контейнер \texttt{LinearLayout} представляет простейший контейнер - объект
\texttt{ViewGroup}, который упорядочивает все дочерние элементы в одном
направлении: по горизонтали или по вертикали.
Все элементы расположены один за другим.
Направление разметки указывается с помощью атрибута
\texttt{android:orientation}.
Если, например, ориентация разметки вертикальная
\texttt{(android:orientation="vertical")}, то все элементы располагаются в столбик - по
одному элементу на каждой строке.
Если ориентация горизонтальная
\texttt{(android:orientation="horizontal")}, то элементы располагаются в одну строку.
Например, расположим элементы в горизонтальный ряд:

\begin{lstlisting}[language=xml, caption=\leftline{xml}, label=lst:LinearLayout]
<?xml version="1.0" encoding="utf-8"?>
<LinearLayout xmlns:android="http://schemas.android.com/apk/res/android"
android:layout_width="match_parent"
android:layout_height="match_parent"
android:orientation="horizontal" >
<TextView
android:layout_width="wrap_content"
android:layout_height="wrap_content"
android:layout_margin="5dp"
android:text="Hello"
android:textSize="26sp" />
<TextView
android:layout_width="wrap_content"
android:layout_height="wrap_content"
android:layout_margin="5dp"
android:text="Android"
android:textSize="26sp" />
<TextView
android:layout_width="wrap_content"
android:layout_height="wrap_content"
android:layout_margin="5dp"
android:text="World"
android:textSize="26sp" />
</LinearLayout>
\end{lstlisting}
\subsection{Вес элемента}
\texttt{LinearLayout} поддерживает такое свойство, как вес элемента, которое
передается атрибутом \texttt{android:layout\_weight}.
Это свойство принимает
значение, указывающее, какую часть оставшегося свободного места
контейнера по отношению к другим объектам займет данный элемент.
Например, если один элемент у нас будет иметь для свойства
\texttt{android:layout\_weight} значение 2, а другой - значение 1, то в сумме они дадут
3, поэтому первый элемент будет занимать 2/3 оставшегося пространства, а
второй - 1/3.
Если все элементы имеют значение \texttt{android:layout\_weight="1"}, то все эти
элементы будут равномерно распределены по всей площади контейнера: