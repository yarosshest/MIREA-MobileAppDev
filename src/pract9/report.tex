
\section*{\LARGE{Введение}}
\addcontentsline{toc}{section}{Введение}

\textbf{Задание}:
\begin{itemize}
	\item Изучить применение адаптеров и списков. Реализовать пример связи
		элемента ListView с источником данных с набором элементов TextView с
		помощью одного из таких адаптеров - класса ArrayAdapter.
	\item Реализовать пример с использованием ресурса string-array и ListView.
	\item Реализовать выбор элемента в ListView.
	\item Реализовать множественный выбор в списке.
	\item Реализовать добавление и удаление в ArrayAdapter и ListView
	\item Реализовать расширение списков и создание адаптера.
	\item Реализовать оптимизацию адаптера и применение View Holder
	\item Реализовать сложный список с кнопками
	\item Реализовать выпадающий список Spinner
	\item Реализовать обработку выбора элемента используя слушатель
		OnItemSelectedListener
	\item Реализовать виджет автодополнения AutoCompleteTextView
	\item Реализовать виджет MultiAutoCompleteTextView.
	\item Реализовать пример с элементом GridView
	\item Реализовать пример с элементом RecyclerView
	\item Реализовать обработку выбора элемента в RecyclerView
\end{itemize}

\clearpage

\section*{\LARGE{Выполнение практической работы}}
\addcontentsline{toc}{section}{Выполнение практической работы}

\section{Адаптеры и списки}
Android представляет широкую палитру элементов, которые представляют
списки. Все они является наследниками класса android.widget.AdapterView.
Это такие виджеты как ListView, GridView, Spinner. Они могут выступать
контейнерами для других элементов управления.\par
При работе со списками мы имеем дело с тремя компонентами. Во-первых,
это визуальный элемент или виджет, который на экране представляет список
(ListView, GridView) и который отображает данные. Во-вторых, это источник
данных - массив, объект ArrayList, база данных и т.д., в котором находятся
сами отображаемые данные. И в-третьих, это адаптер - специальный
компонент, который связывает источник данных с виджетом списка.
Одним из самых простых и распространенных элементов списка является
виджет ListView. Рассмотрим связь элемента ListView с источником данных
с помощью одного из таких адаптеров --- класса ArrayAdapter.\par
Класс ArrayAdapter представляет собой простейший адаптер, который
связывает массив данных с набором элементов TextView, из которых, к
примеру, может состоять ListView. То есть в данном случае источником
данных выступает массив объектов. ArrayAdapter вызывает у каждого
объекта метод \texttt{toString()} для приведения к строковому виду
и полученную строку устанавливает в элемент TextView.\par
Пример разметки приложения показан в листниге~\ref{lst:xml:listvw}

\begin{lstlisting}[language=XML
	, label=lst:xml:listvw
	]
<TextView
	android:id="@+id/selection"
	android:layout_width="0dp"
	android:layout_height="wrap_content"
	android:textSize="30sp"
	app:layout_constraintLeft_toLeftOf="parent"
	app:layout_constraintRight_toRightOf="parent"
	app:layout_constraintTop_toTopOf="parent" />

<ListView
	android:id="@+id/list"

	android:layout_width="match_parent"
	android:layout_height="0dp"

	android:entries="@array/jobs"
	app:layout_constraintLeft_toLeftOf="parent"
	app:layout_constraintRight_toRightOf="parent"
	app:layout_constraintTop_toBottomOf="@+id/selection">
</ListView>
\end{lstlisting}

\section{string-array}
Чтобы не задавать массив строк программно в коде java,
подобный массив гораздо удобнее хранить в файле xml в виде ресурса.
Ресурсы массивов строк представляют элемент типа string-array. Они могут
находится в каталоге res/values в xml-файле с произвольным именем.\par
Массив строк задается с помощью элемента \texttt{<string-array>}, атрибут
name которого может иметь произвольное значение, по которому затем будут
ссылаться на этот массив. Все элементы массива представляют набор
значений \texttt{<item>}.\par
Пример содержимого ресурса показан в листниге~\ref{lst:xml:string-array}.

\begin{lstlisting}[language=XML
	, label=lst:xml:string-array
	]
<?xml version="1.0" encoding="utf-8"?>
<resources>
    <string-array name="jobs">
		<item> Accounting technician </item>
		<item> Fashion designer </item>
		<item> Engineer, manufacturing systems </item>
		<item> Development worker, international aid </item>
		<item> Research scientist (life sciences) </item>
		<item> Therapist, drama </item>
		<item> Secretary/administrator </item>
    </string-array>
</resources>
\end{lstlisting}

Тодгда получить ресурс массив строк можно кодом:

\begin{verbatim}
String[] countries = getResources().getStringArray(R.array.countries);
\end{verbatim}

\section{Выбор элемента в ListView}
Кроме простого вывода списка элементов ListView позволяет выбирать
элемент и обрабатывать его выбор.\par
Для этого свяжем список ListView с источником данных и закрепим за ним
слушатель нажатия на элемент списка, как показано
в листниге~\ref{lst:list:select}.

\begin{lstlisting}[language=Java
	, label=lst:list:select
	]
TextView selection = findViewById(R.id.selection);
ListView listView = findViewById(R.id.list);

String[] jobs2 = getResources().getStringArray(R.array.jobs);
ArrayAdapter<String> adapter = new ArrayAdapter(this,
		android.R.layout.simple_list_item_1, jobs2);
listView.setAdapter(adapter);

listView.setOnItemClickListener(new AdapterView.OnItemClickListener() {
	@Override
	public void onItemClick(AdapterView<?> parent
		, View view, int position, long id) {
		String selectedItem = jobs2[position];
		selection.setText(selectedItem);
	}
 });
\end{lstlisting}

\section{Реализовать множественный выбор в списке}
Иногда требуется выбрать не один элемент, как по умолчанию, а несколько.
Для этого в разметке списка надо установить атрибут
\texttt{android:choiceMode="multipleChoice"}.\par
А определение обработки выбора показана в листниге~\ref{lst:list:multy-select}

\begin{lstlisting}[language=Java
	, label=lst:list:select
	]
TextView selection2 = findViewById(R.id.selection2);
ListView listView2 = findViewById(R.id.list2);
ArrayAdapter<String> adapter2 = new ArrayAdapter(this,
		android.R.layout.simple_list_item_multiple_choice, jobs2);
listView2.setAdapter(adapter2);
listView2.setOnItemClickListener(new AdapterView.OnItemClickListener() {
	 @Override
	 public void onItemClick(AdapterView<?> parent
		, View v
		, int position
		, long id)
	 {
		 SparseBooleanArray selected = 
			listView2.getCheckedItemPositions();
		 String selectedItems = "";
		 for(int i = 0; i < jobs2.length; i++) {
			 if(selected.get(i))
				 selectedItems += jobs2[i] + ",";
		 }
		 selection2.setText("Selected: " + selectedItems);
	 }
 });
\end{lstlisting}

\section{Добавление и удаление в ArrayAdapter и ListView}
После привязки ListView к источнику данных через адаптер мы можем
работать с данными - добавлять, удалять, изменять только через адаптер.
ListView служит только для отображения данных.\par
Для управления данными мы можем использовать методы адаптера или
напрямую источника данных. Например, класс ArrayAdapter предоставляет
следующие методы для управления данными:

\begin{itemize}
	\item \texttt{void add(T object)}: добавляет элемент object
		в конец массива;
	\item \texttt{void addAll(T... items)}: добавляет все элементы items
		в конец массива;
	\item \texttt{void addAll(Collection<? extends T> collection)}:
		добавляет коллекцию элементов collection в конец массива;
	\item \texttt{void clear()}: удаляет все элементы из списка;
	\item \texttt{void insert(T object, int index)}: добавляет элемент
		object в массив по индексу index;
	\item \texttt{void remove(T object)}: удаляет элемент object из массива.
\end{itemize}

Однако после применения вышеуказанных методов изменения коснутся
только массива, выступающего источником данных. Чтобы
синхронизировать изменения с элементом ListView, надо вызвать у адаптера
метод \texttt{notifyDataSetChanged()}.\par
Для вывода списка в xml-файле предназначен ListView с возможностью
множественного выбора элементов. Для добавления и удаления определены
две кнопки. Для ввода нового объекта в список предназначено поле EditText.\par
Код для управления списком показан в листниге~\ref{lst:list:chng}.

\begin{lstlisting}[language=Java
	, label=lst:list:chng
	]
ArrayList<String> users = new ArrayList<String>();
ArrayList<String> selectedUsers = new ArrayList<String>();
ArrayAdapter<String> adapter;
ListView usersList;

@Override
protected void onCreate(Bundle savedInstanceState) {
	super.onCreate(savedInstanceState);
	setContentView(R.layout.activity_chng_list);
	Collections.addAll(users, "Katherine", "Lauren", "David"
		, "Aaron", "April", "Catherine", "Chad")
	usersList = findViewById(R.id.usersList);
	adapter = new ArrayAdapter(this,
			android.R.layout.simple_list_item_multiple_choice, users);
	usersList.setAdapter(adapter);
	usersList.setOnItemClickListener(new AdapterView.OnItemClickListener(){
		@Override
		public void onItemClick(AdapterView<?> parent, View v, int position,
								long id)
		{
			String user = adapter.getItem(position);
			if(usersList.isItemChecked(position))
				selectedUsers.add(user);
			else
				selectedUsers.remove(user);
		}
	});
}
public void add(View view){
	EditText userName = findViewById(R.id.userName);
	String user = userName.getText().toString();
	if(!user.isEmpty()){
		adapter.add(user);
		userName.setText("");
		adapter.notifyDataSetChanged();
	}
}
public void remove(View view){
	for(int i=0; i< selectedUsers.size();i++){
		adapter.remove(selectedUsers.get(i));
	}
	usersList.clearChoices();
	selectedUsers.clear();
	adapter.notifyDataSetChanged();
}
\end{lstlisting}

\section{Реализовать расширение списков и создание адаптера}
Традиционные списки ListView, использующие стандартные адаптеры
ArrayAdapter, прекрасно работают с массивами строк. Однако чаще мы будем
сталкиваться с более сложными по структуре списками, где один элемент
представляет не одну строку, а несколько строк, картинок и других
компонентов.\par
Для создания сложного списка нам надо переопределить один из
используемых адаптеров. Поскольку, как правило, используется
ArrayAdapter, то именно его мы и переопределим.
Его код проиллюстрирован в листниге~\ref{lst:adapter:list:state}.

\begin{lstlisting}[language=Java
	, label=lst:adapter:list:state
	]
public class StateAdapter extends ArrayAdapter<State> {
	private LayoutInflater inflater;
	private int layout;
	private List<State> states;

	public StateAdapter(Context context, int resource, List<State> states) {
		super(context, resource, states);
		this.states = states;
		this.layout = resource;
		this.inflater = LayoutInflater.from(context);
	}

	public View getView(int position, View convertView, ViewGroup parent) {
		ViewHolder viewHolder;
		if(convertView == null){
			convertView = inflater.inflate(this.layout, parent, false);
			viewHolder = new ViewHolder(convertView);
			convertView.setTag(viewHolder);
		}
		else{
			viewHolder = (ViewHolder) convertView.getTag();
		}
		State state = states.get(position);
		viewHolder.imageView.setImageResource(state.getFlagResource());
		viewHolder.nameView.setText(state.getName());
		viewHolder.capitalView.setText(state.getCapital());
		return convertView;
	}

	static class ViewHolder {
		ImageView imageView;
		TextView nameView, capitalView;

		ViewHolder(View view){
			imageView = view.findViewById(R.id.flag);
			nameView = view.findViewById(R.id.name);
			capitalView = view.findViewById(R.id.capital);
		}
	}
}
\end{lstlisting}

Также определим модель, данные которой будут отображаться в списке
(Листнинг~\ref{lst:model:state}).

\begin{lstlisting}[language=Java
	, label=lst:model:state
	]
public class State {
	private String name;
	private String capital;
	private int flagResource;

	public State(String name, String capital, int flagResource) {
		this.name = name;
		this.capital = capital;
		this.flagResource = flagResource;
	}

	public String getName() {
		return name;
	}

	public void setName(String name) {
		this.name = name;
	}

	public String getCapital() {
		return capital;
	}

	public void setCapital(String capital) {
		this.capital = capital;
	}

	public int getFlagResource() {
		return flagResource;
	}

	public void setFlagResource(int flagResource) {
		this.flagResource = flagResource;
	}
}
\end{lstlisting}

Теперь осталось в классе активити соединить адаптер и ListView,
как показано в листниге~\ref{lst:list:expansion}.
\begin{lstlisting}[language=Java
	, label=lst:list:expansion
	]
@Override
protected void onCreate(Bundle savedInstanceState) {
	super.onCreate(savedInstanceState);
	setContentView(R.layout.activity_states_list);
	setInitialData();
	countriesList = findViewById(R.id.state_list);
	StateAdapter stateAdapter = new StateAdapter(this
		, R.layout.state_list_item, states);
	countriesList.setAdapter(stateAdapter);
	AdapterView.OnItemClickListener itemListener =
		new AdapterView.OnItemClickListener() {
		@Override
		public void onItemClick(AdapterView<?> parent
			, View v, int position, long id) {
			State selectedState = (State)parent.getItemAtPosition(position);
			Toast.makeText(getApplicationContext(), "Was selected point " +
							selectedState.getName(),
					Toast.LENGTH_SHORT).show();
		}
	};
	countriesList.setOnItemClickListener(itemListener);
}

private void setInitialData(){
	states.add(new State ("Mexico", "Mexico", R.drawable.mx));
	states.add(new State ("Russia", "Russia", R.drawable.ru));
	states.add(new State ("UK"
		, "UK", R.drawable.uk));
	states.add(new State ("Norvegia", "Norvegia", R.drawable.no));
	states.add(new State ("German", "German", R.drawable.de));
}
\end{lstlisting}

\section{Оптимизация адаптера и применение View Holder}
Адаптер, написанный в прошлой главе, имеет один очень большой минус~---
при прокрутке в ListView, если в списке очень много объектов,
то для каждого элемента,
когда он попадет в зону видимости, будет повторно вызываться метод
getView, в котором будет заново создаваться новый объект View.
Соответственно будет увеличиваться потребление памяти и снижаться
производительность. Поэтому оптимизируем код метода getView,
как показано в листниге~\ref{lst:adapter:update}.

\begin{lstlisting}[language=Java
	, label=lst:adapter:update
	]
public View getView(int position, View convertView, ViewGroup parent) {
	ViewHolder viewHolder;
	if(convertView == null){
		convertView = inflater.inflate(this.layout, parent, false);
		viewHolder = new ViewHolder(convertView);
		convertView.setTag(viewHolder);
	} else {
		viewHolder = (ViewHolder) convertView.getTag();
	}
	State state = states.get(position);
	viewHolder.imageView.setImageResource(state.getFlagResource());
	viewHolder.nameView.setText(state.getName());
	viewHolder.capitalView.setText(state.getCapital());
	return convertView;
}
\end{lstlisting}

Параметр convertView указывает на элемент View, который используется для
объекта в списке по позиции position. Если ранее уже создавался View для
этого объекта, то параметр convertView уже содержит некоторое значение,
которое мы можем использовать.\par
В методе getView, если convertView равен null (то есть если ранее для
объекта не создана разметка) создаем объект ViewHolder, который сохраняем
в тег в convertView:

\begin{verbatim}
convertView.setTag(viewHolder);
\end{verbatim}

Если же разметка для объекта в ListView уже ранее была создана, то обратно
получаем ViewHolder из тега:

\begin{verbatim}
viewHolder = (ViewHolder) convertView.getTag();
\end{verbatim}

Затем также для ImageView и TextView во ViewHolder устанавливаются
значения из объекта State:

\begin{verbatim}
viewHolder.imageView.setImageResource(state.getFlagResource());
viewHolder.nameView.setText(state.getName());
viewHolder.capitalView.setText(state.getCapital());
\end{verbatim}

И теперь ListView особенно при больших списках будет работать плавнее и
производительнее, чем в предыдущих материалах.

\section{Сложный список с кнопками}
Ранее были расмотрены кастомные адаптеры, которые позволяют выводить в
списки сложные данные. Теперь пойдем дальше и рассмотрим, как мы можем
добавить в списки другие элементы, например, кнопки, и обрабатывать их
события.\par
Для этого вначале определим класс Product, как показано
в листниге~\ref{lst:model:product}.

\begin{lstlisting}[language=Java
	, label=lst:model:product
	]
public class Product {
	private String name;
	private int count;
	private String unit;

	public Product(String name, String unit) {
		this.name = name;
		this.count = 0;
		this.unit = unit;
	}

	public String getName() {
		return name;
	}

	public void setName(String name) {
		this.name = name;
	}

	public int getCount() {
		return count;
	}

	public void setCount(int count) {
		this.count = count;
	}

	public String getUnit() {
		return unit;
	}

	public void setUnit(String unit) {
		this.unit = unit;
	}
}
\end{lstlisting}

Данный класс хранит название, количество продукта, а также единицу
измерения. И объекты этого классы будем выводить в список.\par
Для этого в каталог res/layout добавим новый файл product\_list\_item.xml,
как показано в листниге~\ref{lst:xml:product-item}.

\begin{lstlisting}[language=XML
	, label=lst:xml:product-item
	]
<?xml version="1.0" encoding="utf-8"?>
<androidx.constraintlayout.widget.ConstraintLayout
    xmlns:android="http://schemas.android.com/apk/res/android"
    xmlns:app="http://schemas.android.com/apk/res-auto"
    android:layout_width="match_parent"
    android:layout_height="wrap_content"
    android:padding="16dp" >
    <TextView
        android:id="@+id/nameView"
        android:layout_width="0dp"
        android:layout_height="wrap_content"
        android:textSize="18sp"
        app:layout_constraintHorizontal_weight="2"
        app:layout_constraintBottom_toBottomOf="parent"
        app:layout_constraintLeft_toLeftOf="parent"
        app:layout_constraintRight_toLeftOf="@+id/countView"
        app:layout_constraintTop_toTopOf="parent"/>
    <TextView
        android:id="@+id/countView"
        android:layout_width="0dp"
        android:layout_height="wrap_content"
        android:textSize="18sp"
        app:layout_constraintHorizontal_weight="2"
        app:layout_constraintBottom_toBottomOf="parent"
        app:layout_constraintLeft_toRightOf="@+id/nameView"
        app:layout_constraintRight_toLeftOf="@+id/addButton"
        app:layout_constraintTop_toTopOf="parent" />
    <Button
        android:id="@+id/addButton"
        android:layout_width="0dp"
        android:layout_height="wrap_content"
        android:text="+"
        app:layout_constraintHorizontal_weight="1"
        app:layout_constraintBottom_toBottomOf="parent"
        app:layout_constraintLeft_toRightOf="@+id/countView"
        app:layout_constraintRight_toLeftOf="@+id/removeButton"
        app:layout_constraintTop_toTopOf="parent" />
    <Button
        android:id="@+id/removeButton"
        android:layout_width="0dp"
        android:layout_height="wrap_content"
        android:text="-"
        app:layout_constraintHorizontal_weight="1"
        app:layout_constraintBottom_toBottomOf="parent"
        app:layout_constraintLeft_toRightOf="@+id/addButton"
        app:layout_constraintRight_toRightOf="parent"
        app:layout_constraintTop_toTopOf="parent"/>
</androidx.constraintlayout.widget.ConstraintLayout>
\end{lstlisting}

Здесь определены два текстовых поля для вывода названия и количества
продукта и две кнопки для добавления и удаления однйо единицы продукта.\par
Теперь добавим класс адаптера, который назовем ProductAdapter, по аналогии
какой был создан в предыдущей главе.\par
И немного изменим управляющее активити, как показано
в листниге~\ref{lst:list:product}.

\begin{lstlisting}[language=Java
	, label=lst:list:product
	]
@Override
protected void onCreate(Bundle savedInstanceState) {
	super.onCreate(savedInstanceState);
	setContentView(R.layout.activity_complex_list);
	ArrayList<Product> products = new ArrayList<>();

	if(products.size() == 0){
		products.add(new Product("Patato", "kg."));
		products.add(new Product("Tea", "item."));
		products.add(new Product("Eage", "item."));
		products.add(new Product("Milk", "l."));
		products.add(new Product("Makaroni", "kg."));
	}

	ListView productList = findViewById(R.id.productList);
	ProductAdapter adapter = new ProductAdapter(this
		, R.layout.product_list_item
		, products);
	productList.setAdapter(adapter);
}
\end{lstlisting}

\section{Spinner}
Spinner представляет собой выпадающий список. Определим в файле
разметки элемент Spinner, как показано
в листниге~\ref{lst:xml:spinner}.

\begin{lstlisting}[language=Java
	, label=lst:xml:spinner
	]
<?xml version="1.0" encoding="utf-8"?>
<androidx.constraintlayout.widget.ConstraintLayout
	xmlns:android="http://schemas.android.com/apk/res/android"
	xmlns:app="http://schemas.android.com/apk/res-auto"
	android:layout_width="match_parent"
	android:layout_height="match_parent"
	android:padding="16dp">
	<Spinner
		android:id="@+id/spinner"
		android:layout_width="wrap_content"
		android:layout_height="wrap_content"
		app:layout_constraintLeft_toLeftOf="parent"
		app:layout_constraintTop_toTopOf="parent" />
</androidx.constraintlayout.widget.ConstraintLayout>
\end{lstlisting}

В качестве источника данных, как и для ListView, для Spinner может служить
простой список или массив, соданный программно, либо ресурс string-array.
Взаимодействие с источником данных также будет идти через адаптер. В
данном случае определим источник программно в виде массива в коде
(рис. \ref{lst:java:spinner}).

\begin{lstlisting}[language=Java
	, label=lst:java:spinner
	]
@Override
protected void onCreate(Bundle savedInstanceState) {
	super.onCreate(savedInstanceState);
	setContentView(R.layout.activity_spinner);

	Spinner spinner = findViewById(R.id.spinner);
	ArrayAdapter<String> adapter = new ArrayAdapter(this
			, android.R.layout.simple_spinner_item
			, getResources().getStringArray(R.array.jobs));

	adapter.setDropDownViewResource(android.R.layout.simple_spinner_dropdown_item);
	spinner.setAdapter(adapter);
}
\end{lstlisting}

\section{Обработка выбора элемента в Spinner}
Используя слушатель OnItemSelectedListener, в частности его метод
\texttt{onItemSelected()}, мы можем обрабатывать выбор элемента из списка.\par
Далее определим в активити для элемента Spinner слушатель
OnItemSelectedListener (рис. \ref{lst:spinner:listner}).

\begin{lstlisting}[language=Java
	, label=lst:spinner:listner
	]
@Override
protected void onCreate(Bundle savedInstanceState) {
	super.onCreate(savedInstanceState);
	setContentView(R.layout.activity_spinner);

	Spinner spinner = findViewById(R.id.spinner);
	ArrayAdapter<String> adapter = new ArrayAdapter(this
			, android.R.layout.simple_spinner_item
			, getResources().getStringArray(R.array.jobs));

	adapter.setDropDownViewResource(android.R.layout.simple_spinner_dropdown_item);
	spinner.setAdapter(adapter);
	TextView selection = findViewById(R.id.selection);
	AdapterView.OnItemSelectedListener itemSelectedListener = new
		AdapterView.OnItemSelectedListener() {
			@Override
			public void onItemSelected(AdapterView<?> parent
				, View view
				, int position
				, long id) {
				String item = (String)parent.getItemAtPosition(position);
				selection.setText(item);
			}
			@Override
			public void onNothingSelected(AdapterView<?> parent) {
			}
		};
	spinner.setOnItemSelectedListener(itemSelectedListener);
}
\end{lstlisting}

\section{Виджет автодополнения AutoCompleteTextView}
AutoCompleteTextView представляет элемент, созданный на основе класса
EditText и обладающий возможностью автодополнения.\par
Во-первых, объявим в ресурсе разметке данный элемент, как показано
в листниге~\ref{lst:xml:autocomplete}.

\begin{lstlisting}[language=XML
	, label=lst:xml:autocomplete
	]
<?xml version="1.0" encoding="utf-8"?>
<androidx.constraintlayout.widget.ConstraintLayout
	xmlns:android="http://schemas.android.com/apk/res/android"
	xmlns:app="http://schemas.android.com/apk/res-auto"
	android:layout_width="match_parent"
	android:layout_height="match_parent"
	android:padding="16dp">
	<AutoCompleteTextView
		android:id="@+id/autocomplete"
		android:layout_width="0dp"
		android:layout_height="wrap_content"
		android:completionHint="Input citi"
		android:completionThreshold="1"
		app:layout_constraintLeft_toLeftOf="parent"
		app:layout_constraintRight_toRightOf="parent"
		app:layout_constraintTop_toTopOf="parent" />
</androidx.constraintlayout.widget.ConstraintLayout>
\end{lstlisting}

Атрибут android:completionHint позволяет задать надпись, которая
отображается внизу списка, а свойство android:completionThreshold
устанавливает, какое количество символов надо ввести, чтобы начало
работать автодополнение. То есть в данном случае уже после ввода одного
символа должен появться список с подстановками.\par
Как и в случае с элементами ListView и Spinner, AutoCompleteTextView
подключается к источнику данных через адаптер. Источником данных опять
же может служить массив или список объектов, либо ресурс string-array.\par
Теперь подключим к виджету массив строк в классе активити
(рис.~\ref{lst:java:autocomplete}).

\begin{lstlisting}[language=Java
	, label=lst:java:autocomplete
	]
AutoCompleteTextView autoCompleteTextView = findViewById(R.id.autocomplete);
ArrayAdapter<String> adapter = new ArrayAdapter(this
		, androidx.appcompat.R.layout.support_simple_spinner_dropdown_item
		, cities);
autoCompleteTextView.setAdapter(adapter);
\end{lstlisting}

\section{виджет MultiAutoCompleteTextView}
Этот виджет дополняет функциональность элемента AutoCompleteTextView.
MultiAutoCompleteTextView позволяет использовать автодополнения не
только для одной строки, но и для отдельных слов. Например, если вводится
слово и после него ставится запятая, то автозаполнение все равно будет
работать для вновь вводимых слов после запятой или другого разделителя.\par
Подключение к виджету массив строк показн в классе активити
(рис.~\ref{lst:java:multyautocomplete}).

\begin{lstlisting}[language=Java
	, label=lst:java:multyautocomplete
	]
MultiAutoCompleteTextView multiAutoCompleteTV = 
		findViewById(R.id.multyautocomplete);
ArrayAdapter<String> adapterM = new ArrayAdapter(this
		, androidx.appcompat.R.layout.support_simple_spinner_dropdown_item
		, cities);
multiAutoCompleteTextView.setAdapter(adapterM);
multiAutoCompleteTextView.setTokenizer(
	new MultiAutoCompleteTextView.CommaTokenizer()
);
\end{lstlisting}

\section{GridView}
Элемент GridView представляет отображение в виде таблицы - набора строк
и столбцов.\par
Основные атрибуты GridView:

\begin{itemize}
	\item \texttt{android:columnWidth}: устанавливает фиксированную ширину
		столбцов;
	\item \texttt{android:gravity}: устанавливает выравнивание содержимого
		внутри каждой ячейки;
	\item \texttt{android:horizontalSpacing}: устанавливает горизональные
		отступы между столбцами;
	\item \texttt{android:numColumns}: устанавливает количество столбцов;
	\item \texttt{android:stretchMode}: устанавливает, как столбцы будут
		растягиваться и занимать пространство контейнера.
\end{itemize}

Определение GridView в разметке проиллюстрирована
в листниге~\ref{lst:xml:gridview}.
\begin{lstlisting}[language=XML
	, label=lst:xml:gridview
	]
<?xml version="1.0" encoding="utf-8"?>
<androidx.constraintlayout.widget.ConstraintLayout
    xmlns:android="http://schemas.android.com/apk/res/android"
    xmlns:app="http://schemas.android.com/apk/res-auto"
    android:layout_width="match_parent"
    android:layout_height="match_parent"
    android:padding="16dp">
    <GridView
        android:id="@+id/grid"
        android:layout_width="0dp"
        android:layout_height="0dp"
        android:numColumns="2"
        android:verticalSpacing="16dp"
        android:horizontalSpacing="16dp"
        android:stretchMode="columnWidth"
        app:layout_constraintLeft_toLeftOf="parent"
        app:layout_constraintRight_toRightOf="parent"
        app:layout_constraintTop_toTopOf="parent"
        app:layout_constraintBottom_toBottomOf="parent" />
</androidx.constraintlayout.widget.ConstraintLayout>
\end{lstlisting}

В данном случае указываем, что грид будет иметь 2 столбца, которые
растягиваются равномерно по всей ширине грида, а между ячейками будут
отступы по горизонтали и вертикали в 16 dp.\par
Теперь, как и в случае с ListView, надо установить связь с адаптером в
классе активит, как показано в листниге~\ref{lst:java:gridview}.

\begin{lstlisting}[language=Java
	, label=lst:java:gridview
	]
String[] countries = { "Brasil", "Argantina", "Chili", "Columbia", "Urugvai"};

@Override
protected void onCreate(Bundle savedInstanceState) {
	super.onCreate(savedInstanceState);
	setContentView(R.layout.activity_grid);

	GridView countriesList = findViewById(R.id.grid);
	ArrayAdapter<String> adapter = new ArrayAdapter(this
			, android.R.layout.simple_list_item_1, countries);
	countriesList.setAdapter(adapter);
	AdapterView.OnItemClickListener itemListener =
		new AdapterView.OnItemClickListener() {
			@Override
			public void onItemClick(AdapterView<?> parent
				, View view, int position, long id) {
				Toast.makeText(getApplicationContext()
						,"You selected "
							+ parent.getItemAtPosition(position).toString()
						, Toast.LENGTH_SHORT).show();
			}
	};
	countriesList.setOnItemClickListener(itemListener);
}
\end{lstlisting}

\section{RecyclerView}
Элемент RecyclerView предназначен для оптимизации работы со списками и
во многом позволяет повысить производительность по сравнению со
стандартным ListView.\par
Для представления данных будем использовать созданную ранее модель State.\par
Как и в случае с ListView, для вывода сложных объектов в RecyclerView
необходимо определить свой адаптер. Поэтому добавим новый класс Java,
который назовем StateAdapterRec.
Его код продемонстрирован в листниге~\ref{lst:adapter:recycler}.

\begin{lstlisting}[language=Java
	, label=lst:adapter:recycler
	]
public class StateAdapterRec
	extends RecyclerView.Adapter<StateAdapterRec.ViewHolder>{
	private final LayoutInflater inflater;
	private final List<State> states;

	public StateAdapterRec(Context context, List<State> states) {
		this.states = states;
		this.inflater = LayoutInflater.from(context);
	}

	@Override
	public StateAdapterRec.ViewHolder onCreateViewHolder(ViewGroup parent, int
			viewType) {
		View view = inflater.inflate(R.layout.state_list_item, parent, false);
		return new ViewHolder(view);
	}

	@Override
	public void onBindViewHolder(@NonNull ViewHolder holder, int position) {
		State state = states.get(position);
		holder.flagView.setImageResource(state.getFlagResource());
		holder.nameView.setText(state.getName());
		holder.capitalView.setText(state.getCapital()); }

	@Override
	public int getItemCount() {
		return states.size();
	}

	public static class ViewHolder extends RecyclerView.ViewHolder {
		final ImageView flagView;
		final TextView nameView, capitalView;

		ViewHolder(View view){
			super(view);
			flagView = view.findViewById(R.id.flag);
			nameView = view.findViewById(R.id.name);
			capitalView = view.findViewById(R.id.capital);
		}
	}
}
\end{lstlisting}

В данном случае с помощью значения
\texttt{androidx.recyclerview.widget.LinearLayoutManager} указываем, что
элементы будут располагаться в виде простого списка. Обратите внимание,
что класс LinearLayoutManager также расположен в библиотеке RecyclerView
и поэтому при указании значения указывается полное название класса с
именем его пакета.\par
И в конце создадим активити с кодом, показанным
в листниге~\ref{lst:java:recycler}.

\begin{lstlisting}[language=Java
	, label=lst:java:recycler
	]
ArrayList<State> states = new ArrayList<>();

@Override
protected void onCreate(Bundle savedInstanceState) {
	super.onCreate(savedInstanceState);
	setContentView(R.layout.activity_recycler);
	setInitialData();
	RecyclerView recyclerView = findViewById(R.id.list);
	StateAdapterRec adapter = new StateAdapterRec(this , states);
	recyclerView.setAdapter(adapter);
}

private void setInitialData() {
	states.add(new State ("Mexico", "Mexico", R.drawable.mx));
	states.add(new State ("Russia", "Russia", R.drawable.ru));
	states.add(new State ("UK"
		, "UK", R.drawable.uk));
	states.add(new State ("Norvegia", "Norvegia", R.drawable.no));
	states.add(new State ("German", "German", R.drawable.de));
}
\end{lstlisting}

\section{обработку выбора элемента в RecyclerView}
При работе с виджетом RecyclerView неизбежно встанет вопрос, а как
обработать выбор элемента в RecyclerView. И тут надо заметить, что в
отличие от других типов виджетов для работы со списками RecyclerView по
умолчанию не предоставляет какого-то специального метода, с помощью
которого можно было бы определить слушатель нажатия на элемент в
списке. Поэтому всю инфраструктуру необходимо определять самому
разработчику. Но, к счастью, в реальности все не так сложно.\par
Перейдем к классу StateAdapter и определим его код, как
показано в листниге~\ref{lst:adapter:recycler:select}.

\begin{lstlisting}[language=Java
	, label=lst:adapter:recycler:select
	]
public class StateAdapterRec extends RecyclerView.Adapter<StateAdapterRec.ViewHolder>{
	private final LayoutInflater inflater;
	private final List<State> states;

	public interface OnStateClickListener {
		void onStateClick(State state, int position);
	}

	private final OnStateClickListener onClickListener;

	public StateAdapterRec(Context context, List<State> states
			, OnStateClickListener onClickListener) {
		this.onClickListener = onClickListener;
		this.states = states;
		this.inflater = LayoutInflater.from(context);
	}

	@Override
	public StateAdapterRec.ViewHolder onCreateViewHolder(ViewGroup parent, int
			viewType) {
		View view = inflater.inflate(R.layout.state_list_item, parent, false);
		return new ViewHolder(view);
	}

	@Override
	public void onBindViewHolder(@NonNull ViewHolder holder, int position) {
		State state = states.get(position);
		holder.flagView.setImageResource(state.getFlagResource());
		holder.nameView.setText(state.getName());
		holder.capitalView.setText(state.getCapital());

		holder.itemView.setOnClickListener((v) -> onClickListener.onStateClick(state, position));
	}

	@Override
	public int getItemCount() {
		return states.size();
	}

	public static class ViewHolder extends RecyclerView.ViewHolder {
		final ImageView flagView;
		final TextView nameView, capitalView;

		ViewHolder(View view){
			super(view);
			flagView = view.findViewById(R.id.flag);
			nameView = view.findViewById(R.id.name);
			capitalView = view.findViewById(R.id.capital);
		}
	}
}
\end{lstlisting}

И в конце изменим класс активит (рис. \ref{lst:java:recycler:select}).

\begin{lstlisting}[language=Java
	, label=lst:java:recycler:select
	]
ArrayList<State> states = new ArrayList<>();

@Override
protected void onCreate(Bundle savedInstanceState) {
	super.onCreate(savedInstanceState);
	setContentView(R.layout.activity_recycler);
	setInitialData();
	RecyclerView recyclerView = findViewById(R.id.list);
	StateAdapterRec.OnStateClickListener stateClickListener =
		new StateAdapterRec.OnStateClickListener() {
			@Override
			public void onStateClick(State state, int position) {
				Toast.makeText(getApplicationContext()
					, "Was selected point " + state.getName()
					, Toast.LENGTH_SHORT
				).show();
			}
		};
	StateAdapterRec adapter = new StateAdapterRec(this, states, stateClickListener);
	recyclerView.setAdapter(adapter);
}

private void setInitialData() {
	states.add(new State ("Mexico", "Mexico", R.drawable.mx));
	states.add(new State ("Russia", "Russia", R.drawable.ru));
	states.add(new State ("UK"
		, "UK", R.drawable.uk));
	states.add(new State ("Norvegia", "Norvegia", R.drawable.no));
	states.add(new State ("German", "German", R.drawable.de));
}
\end{lstlisting}

\clearpage

\section*{\LARGE{Вывод}}
\addcontentsline{toc}{section}{Вывод}
В этой практической работе были получены знания об создании и управлении
такими списками как:

\begin{itemize}
	\item ListView
	\item Spinner
	\item GridView
	\item RecyclerView
\end{itemize}

Научились создавать адаптеры, для храниния данный в списках и управления
нажатием на элемнт списка.\par
Также познакомились с виджетами организующие автодополнение.

