Ссылка на репозиторий в GitHub c приложением, реализованным в курсовой работе:
\url{https://github.com/yarosshest/MIREA_Kurs_4_sem}

Ссылка на репозиторий в GitHub c серверной частью, реализованным в курсовой работе:
\url{https://github.com/yarosshest/RecomApi}


\begin{lstlisting}[language=Kotlin, caption=\leftline{RetrofitHelper}, label=lst:RetrofitHelper]
object RetrofitHelper {

    private const val BASE_URL = "http://172.31.128.1:8031/"
    private val client: OkHttpClient = OkHttpClient.Builder()
        .connectTimeout(5, TimeUnit.MINUTES)
        .readTimeout(5, TimeUnit.MINUTES)
        .writeTimeout(5, TimeUnit.MINUTES)
        .cookieJar(MyCookieJar())
        .build()

    fun getInstance(): Retrofit {


        return  Retrofit.Builder()
            .baseUrl(BASE_URL)
            .client(client)
            .addConverterFactory(GsonConverterFactory.create())
            .build()
    }
}
\end{lstlisting}

\begin{lstlisting}[language=Kotlin, caption=\leftline{Api}, label=lst:Api]
interface Api {

    @GET("register")
    fun register(
        @Query("log") login: String,
        @Query("password") password: String,
        ): Call<Map<String,Int>>

    @GET("login")
    fun login(
        @Query("log") login: String,
        @Query("password") password: String,
    ): Call<Map<String,Int>>

    @GET("find")
    fun find(
        @Query("line") line: String
    ): Call<List<Film>>

    @GET("get_film")
    fun get_film(
        @Query("prod_id") id: Int
    ): Call<Film>

    @POST("rate")
    fun rate(
        @Query("prod_id") id: Int,
        @Query("user_rate") rate: Boolean,
    ): Call<String>


    @GET("get_recommendations")
    fun get_recommendations(
    ): Call<List<Film>>

}
\end{lstlisting}

\begin{lstlisting}[language=Kotlin, caption=\leftline{Register}, label=lst:Register]
class Register : AppCompatActivity() {
    override fun onCreate(savedInstanceState: Bundle?) {
        super.onCreate(savedInstanceState)
        setContentView(R.layout.activity_register)
    }

    fun onClickRegister(view: View) {

        val intent = Intent(this, Login::class.java)

        val textLogin = findViewById<EditText>(R.id.edittTextTextLogin)
        val textPassword = findViewById<EditText>(R.id.editTextTextPassword)
        val textError = findViewById<TextView>(R.id.textError)

        val api = RetrofitHelper.getInstance().create(Api::class.java)
        val call = api.register(
            login = textLogin.text.toString(),
            password = textPassword.text.toString()
        )

        call.enqueue(object : Callback<Map<String, Int>>{
            override fun onResponse(
                call: Call<Map<String, Int>>,
                response: Response<Map<String, Int>>
            ) {
                if (response.isSuccessful) {
                    startActivity(intent)
                }else{
                    if (response.code() == 405)
                        textError.text = "User already exists"
                }
            }

            override fun onFailure(call: Call<Map<String, Int>>, t: Throwable) {
                println("Network Error :: " + t.localizedMessage);
            }

        })

    }
}
\end{lstlisting}

\begin{lstlisting}[language=Kotlin, caption=\leftline{Login}, label=lst:Login]
class Login : AppCompatActivity() {
    override fun onCreate(savedInstanceState: Bundle?) {
        super.onCreate(savedInstanceState)
        setContentView(R.layout.activity_login)
    }

    fun onLoginClick(view: View) {
        val intent = Intent(this, MainAppActivity::class.java)

        val textLogin = findViewById<EditText>(R.id.edittTextTextLogin)
        val textPassword = findViewById<EditText>(R.id.editTextTextPassword)
        val textError = findViewById<TextView>(R.id.textError)

        val api = RetrofitHelper.getInstance().create(Api::class.java)
        val call = api.login(
            login = textLogin.text.toString(),
            password = textPassword.text.toString()
        )

        call.enqueue(object : Callback<Map<String, Int>> {
            override fun onResponse(
                call: Call<Map<String, Int>>,
                response: Response<Map<String, Int>>
            ) {
                if (response.isSuccessful) {
                    startActivity(intent)
                }else{
                    if (response.code() == 404)
                        textError.text = "User not found"
                }
            }

            override fun onFailure(call: Call<Map<String, Int>>, t: Throwable) {
                println("Network Error :: " + t.localizedMessage);
            }

        })
    }


    fun onClickRegister(view: View) {
        val intent = Intent(this, Register::class.java)
        startActivity(intent)
    }
}
\end{lstlisting}

\begin{lstlisting}[language=Kotlin, caption=\leftline{MainAppActivity}, label=lst:MainAppActivity]
class MainAppActivity : AppCompatActivity() {

    override fun onCreate(savedInstanceState: Bundle?) {
        super.onCreate(savedInstanceState)
        setContentView(R.layout.activity_main_app)

        val navHostFragment =
            supportFragmentManager.findFragmentById(R.id.fragmentContainerView) as NavHostFragment
        val navController = navHostFragment.navController

        val bnv = findViewById<BottomNavigationView >(R.id.bottomNavigationView)


        bnv.setOnItemSelectedListener {
            when(it.itemId) {
                R.id.search -> {
                    navController.navigate(R.id.findFragment )
                    return@setOnItemSelectedListener true
                }

                R.id.recommendation -> {
                    navController.navigate(R.id.recommendationFragment)
                    return@setOnItemSelectedListener true
                }

                R.id.timer -> {
                    navController.navigate(R.id.timerFragment)
                    return@setOnItemSelectedListener true
                }
                else -> {false}
            }
        }
    }
}
\end{lstlisting}

\begin{lstlisting}[language=Kotlin, caption=\leftline{FindFragment}, label=lst:FindFragment]
class FindFragment : Fragment() {
    private lateinit var saveAdapter : FilmAdapter

    private val viewModel: AppViewModel by activityViewModels()

    private lateinit var listFilm : RecyclerView

    override fun onCreateView(
        inflater: LayoutInflater, container: ViewGroup?,
        savedInstanceState: Bundle?
    ): View? {
        return inflater.inflate(R.layout.fragment_find, container, false)
    }

    override fun onViewCreated(view: View, savedInstanceState: Bundle?) {
        super.onViewCreated(view, savedInstanceState)

        val textError : TextView = view.findViewById(R.id.textError)

        val search = view.findViewById<SearchView>(R.id.searchName)

        listFilm = view.findViewById(R.id.listFilm)

        val savedFilms = viewModel.getFindList()


        listFilm.layoutManager = GridLayoutManager(context, 3)

        if (this::saveAdapter.isInitialized){
            saveAdapter.setOnClickListener(object :
                FilmAdapter.OnClickListener {
                override fun onClick(position: Int, model: Film) {
                    val bundle = bundleOf("id_film" to model.id)
                    view.findNavController().navigate(
                        R.id.action_findFragment_to_productFragment,
                        bundle
                    )
                }
            })
            listFilm.adapter = saveAdapter
        } else if (savedFilms != null){
            val adapter = createAdapter(savedFilms)
            listFilm.adapter = adapter
        }

        val api = RetrofitHelper.getInstance().create(Api::class.java)

        search.setOnQueryTextListener(object : SearchView.OnQueryTextListener{
            override fun onQueryTextSubmit(p0: String?): Boolean {
                p0?.let { api.find(line = it) }
                    ?.enqueue(object : Callback<List<Film>> {
                        override fun onResponse(
                            call: Call<List<Film>>,
                            response: Response<List<Film>>
                        ) {
                            if (response.isSuccessful) {
                                textError.visibility = View.GONE
                                val films = response.body()
                                if (films != null) {
                                    val adapter = createAdapter(films)
                                    listFilm.adapter = adapter
                                    saveAdapter = adapter
                                    viewModel.saveFindList(films)
                                }
                            }else if (response.code() == 404){
                                textError.text = "Films not found"
                                textError.visibility = View.VISIBLE
                            }
                        }

                        override fun onFailure(call: Call<List<Film>>, t: Throwable) {
                            println("Network Error :: " + t.localizedMessage);
                        }

                    })
                return true
            }

            override fun onQueryTextChange(p0: String?): Boolean {
                return true
            }
        })
    }

    private fun createAdapter(films : List<Film>) : FilmAdapter{
        val adapter = FilmAdapter(films)
        adapter.setOnClickListener(object :
            FilmAdapter.OnClickListener {
            override fun onClick(position: Int, model: Film) {
                val bundle = bundleOf("id_film" to model.id)
                view?.findNavController()?.navigate(
                    R.id.action_findFragment_to_productFragment,
                    bundle
                )
            }
        })
        return adapter
    }
}
\end{lstlisting}

\begin{lstlisting}[language=Kotlin, caption=\leftline{RecommendationFragment}, label=lst:RecommendationFragment]
class RecommendationFragment : Fragment() {
    private lateinit var saveAdapter : FilmAdapter


    private val viewModel: AppViewModel by activityViewModels()

    private lateinit var binding: FragmentRecommendationBinding

    private lateinit var getButton: Button
    private lateinit var bar: ProgressBar
    private lateinit var textError: TextView
    private lateinit var listFilm: RecyclerView


    override fun onCreate(savedInstanceState: Bundle?) {
        super.onCreate(savedInstanceState)

        binding = FragmentRecommendationBinding.inflate(layoutInflater)

        getButton  = binding.buttonRecommindations
        bar = binding.progressBar
        textError = binding.textError
        listFilm = binding.listFilm
    }


    override fun onCreateView(
        inflater: LayoutInflater, container: ViewGroup?,
        savedInstanceState: Bundle?
    ): View {
        return binding.root
    }

    override fun onViewCreated(view: View, savedInstanceState: Bundle?) {
        super.onViewCreated(view, savedInstanceState)

        listFilm.layoutManager = GridLayoutManager(context, 3)


        val savedFilms = viewModel.getRecommendList()


        val recyclerViewObserver = Observer<List<Film>> {
            val adapter = createAdapter(it)
            listFilm.adapter = adapter
            getButton.visibility = View.VISIBLE
            bar.visibility = View.GONE
        }


        viewModel.getObserveRecommendList().observe(viewLifecycleOwner, recyclerViewObserver)

        val searching = viewModel.getStatusRecommendService()

        if (searching == true) {
            getButton.visibility = View.GONE
            bar.visibility = View.VISIBLE
        }


        if (this::saveAdapter.isInitialized){
            saveAdapter.setOnClickListener(object :
                FilmAdapter.OnClickListener {
                override fun onClick(position: Int, model: Film) {
                    val bundle = bundleOf("id_film" to model.id)
                    view.findNavController().navigate(
                        R.id.action_recommendationFragment_to_productFragment,
                        bundle
                    )
                }
            })
            listFilm.adapter = saveAdapter
        }else if (savedFilms != null){
            val adapter = createAdapter(savedFilms)
            listFilm.adapter = adapter
        }


        getButton.setOnClickListener {
            viewModel.saveStatusRecommendService(true)
            bar.visibility = View.VISIBLE
            getButton.visibility = View.GONE

            val api = RetrofitHelper.getInstance().create(Api::class.java)
            val call = api.get_recommendations()

            call.enqueue(object : Callback<List<Film>>{
                override fun onResponse(call: Call<List<Film>>, response: Response<List<Film>>) {
                    viewModel.saveStatusRecommendService(false)
                    bar.visibility = View.GONE
                    getButton.visibility = View.VISIBLE
                    if (response.isSuccessful) {
                        val films = response.body()
                        if (films != null) {
                            val adapter = createAdapter(films)
                            listFilm.adapter = adapter
                            saveAdapter = adapter
                            viewModel.saveRecommendList(films)
                        }
                    } else if (response.code() == 405) {
                        textError.text = "Less 2 positive rates"
                        textError.visibility = View.VISIBLE
                    }else if (response.code() == 406) {
                        textError.text = "Less 2 negative rates"
                        textError.visibility = View.VISIBLE
                    }
                }

                override fun onFailure(call: Call<List<Film>>, t: Throwable) {
                    textError.text = "Connection lost"
                    textError.visibility = View.VISIBLE
                    println("Network Error :: " + t.localizedMessage);
                }
            })
        }
    }

     fun createAdapter(films : List<Film>) : FilmAdapter{
        val adapter = FilmAdapter(films)
        adapter.setOnClickListener(object :
            FilmAdapter.OnClickListener {
            override fun onClick(position: Int, model: Film) {
                val bundle = bundleOf("id_film" to model.id)
                view?.findNavController()?.navigate(
                    R.id.action_recommendationFragment_to_productFragment,
                    bundle
                )
            }
        })
        return adapter
    }

}
\end{lstlisting}

\begin{lstlisting}[language=Kotlin, caption=\leftline{TimerFragment}, label=lst:TimerFragment]
class TimerFragment : Fragment() {

    private val TAG = "TimerFragment"

    private lateinit var binding: FragmentTimerBinding

    private lateinit var button : FloatingActionButton
    private lateinit var clock : TextView
    private var timeTimer : Long = 30*60000
    private var play : Boolean = false

    private var settings: SharedPreferences? = null

    companion object {
        const val preferences = "preferences"
        const val TIME = "time"
        const val STATUS = "play"
        const val NOTIFICATION_ID = 101
        const val CHANNEL_ID = "1"
    }

    override fun onCreate(savedInstanceState: Bundle?) {
        super.onCreate(savedInstanceState)

        binding = FragmentTimerBinding.inflate(layoutInflater)

        clock  = binding.textView
        button = binding.imageButton

    }

    override fun onCreateView(
        inflater: LayoutInflater, container: ViewGroup?,
        savedInstanceState: Bundle?
    ): View {
        return binding.root
    }

    override fun onViewCreated(view: View, savedInstanceState: Bundle?) {
        super.onViewCreated(view, savedInstanceState)


        clock.setOnClickListener(onTimeClicklistener)
        button.setOnClickListener(onPlayClick)

    }



    private val broadcastReceiver: BroadcastReceiver = object : BroadcastReceiver() {
        override fun onReceive(context: Context, intent: Intent) {
            //Update GUI
            updateGUI(intent)
        }
    }

    override fun onResume() {
        super.onResume()
        play = requireActivity().getSharedPreferences(
            preferences, Context.MODE_PRIVATE).getBoolean(STATUS, false)
        if (play){
            button.setImageResource(R.drawable.baseline_pause_circle_outline)
        }else{
            button.setImageResource(R.drawable.baseline_play_circle_outline)
        }

        timeTimer = requireActivity().getSharedPreferences(
            preferences, Context.MODE_PRIVATE).getLong(TIME, 30000)
        clock.text = convertSecondsToHMmSs(timeTimer)
        context?.registerReceiver(broadcastReceiver, IntentFilter(TimerService.COUNTDOWN_BR))
    }

    override fun onPause() {
        super.onPause()
        requireActivity().getSharedPreferences(
            preferences, Context.MODE_PRIVATE).edit().putBoolean(STATUS, play).apply()
        requireActivity().getSharedPreferences(
            preferences, Context.MODE_PRIVATE).edit().putLong(
            TIME, timeTimer).apply()
        context?.unregisterReceiver(broadcastReceiver)
    }

    override fun onStop() {
        try {
            context?.unregisterReceiver(broadcastReceiver)
        } catch (e: Exception) {
            // Receiver was probably already
        }
        super.onStop()
    }

    override fun onDestroy() {
        context?.stopService(Intent(context, TimerService::class.java))
        super.onDestroy()
    }




    private fun updateGUI(intent: Intent) {
        if (intent.extras != null) {
            val millisUntilFinished : Long = intent.getLongExtra("countdown", 30000)
            clock.text = convertSecondsToHMmSs(millisUntilFinished)
            timeTimer = millisUntilFinished
        }
    }




    private val onPlayClick = View.OnClickListener { view ->
        when (view.id) {
            R.id.imageButton -> {
                val button : FloatingActionButton = view.findViewById(R.id.imageButton)
                play = if(!play) {
                    val intent = Intent(context, TimerService::class.java)
                    intent.putExtra("time", timeTimer)
                    context?.startService(intent)
                    Log.i(TAG,"Started Service")
                    button.setImageResource(R.drawable.baseline_pause_circle_outline)
                    true
                }else{
                    context?.stopService(Intent(context, TimerService::class.java))
                    button.setImageResource(R.drawable.baseline_play_circle_outline)
                    false
                }
            }
        }
    }


    private val onTimeClicklistener= View.OnClickListener { view ->
        when (view.id) {
            R.id.textView -> {
                play = false
                button.setImageResource(R.drawable.baseline_play_circle_outline)
                context?.stopService(Intent(context, TimerService::class.java))
                val picker =
                    MaterialTimePicker.Builder()
                        .setTimeFormat(TimeFormat.CLOCK_24H)
                        .setTitleText("Select Appointment time")
                        .setInputMode(INPUT_MODE_CLOCK)
                        .build()
                fragmentManager?.let { picker.show(it, "tag") }

                picker.addOnPositiveButtonClickListener {
                    val newHour: Int = picker.hour
                    val newMinute: Int = picker.minute
                    val time: String = java.lang.String.format(
                        Locale.getDefault(),
                        "%02d:%02d",
                        newHour,
                        newMinute
                    )
                    clock.text = time
                    timeTimer = (newHour*3600000).toLong()
                    timeTimer += (newMinute*60000).toLong()

                }
            }
        }
    }

    private fun convertSecondsToHMmSs(milliseconds: Long): String {
        val s = (milliseconds/1000) % 60
        val m = milliseconds / 60000  % 60
        val h = milliseconds / 3600000  % 24
        return "%d:%02d:%02d".format(h, m, s)
    }


}
\end{lstlisting}

\begin{lstlisting}[language=Kotlin, caption=\leftline{TimerService}, label=lst:TimerService]
class TimerService : Service() {
    private lateinit var builder : NotificationCompat.Builder
    private lateinit var notificationManager : NotificationManagerCompat

    private lateinit var timer : CountDownTimer

    companion object {
        const val COUNTDOWN_BR = "com.example.mirea_kurs_4_sem.timer"
        val intent = Intent(COUNTDOWN_BR)
        const val NOTIFICATION_ID = 101
        const val CHANNEL_ID = "1"
        private val TAG = "TimerService"
    }


    override fun onStartCommand(intent: Intent?, flags: Int, startId: Int): Int {
        val intentNotification = Intent(applicationContext, TimerFragment::class.java)
        val pendingIntent: PendingIntent = PendingIntent.getActivity(
            applicationContext, 0, intentNotification, PendingIntent.FLAG_IMMUTABLE)


        builder = applicationContext?.let {
            NotificationCompat.Builder(it, CHANNEL_ID)
                .setSmallIcon(R.drawable.baseline_timer)
                .setContentTitle("Напоминание")
                .setContentText("Время просмотра законченно")
                .setAutoCancel(true)
                .setPriority(NotificationCompat.PRIORITY_DEFAULT)
                .setOngoing(true)
                .setContentIntent(pendingIntent)
        }!!
        notificationManager = applicationContext?.let { NotificationManagerCompat.from(it) }!!
        createNotificationChannel()
        val millis : Long = intent!!.getLongExtra("time", 3000)


        timer = object : CountDownTimer(millis, 100){
            override fun onTick(millisUntilFinished: Long) {
                Companion.intent.putExtra("countdown",millisUntilFinished);
                sendBroadcast(Companion.intent);
            }

            override fun onFinish() {
                createNotificationChannel()
                Companion.intent.putExtra("countdown",0)
                sendBroadcast(Companion.intent);
                with(applicationContext.let { NotificationManagerCompat.from(it) }) {
                    if (applicationContext.let {
                            androidx.core.app.ActivityCompat.checkSelfPermission(
                                it,
                                android.Manifest.permission.POST_NOTIFICATIONS
                            )
                        } != android.content.pm.PackageManager.PERMISSION_GRANTED
                    ) {
                        return
                    }
                    this.notify(
                        TimerFragment.NOTIFICATION_ID,
                        builder.build()
                    )
                }
            }
        }
        timer.start()
        return START_STICKY
    }


    override fun onDestroy() {
        super.onDestroy()
        timer.cancel()
    }



    private fun createNotificationChannel() {
        val name = "Напоминание"
        val descriptionText = "Время просмотра законченно"
        val importance = NotificationManager.IMPORTANCE_DEFAULT
        val channel = NotificationChannel(TimerFragment.CHANNEL_ID, name, importance).apply {
            description = descriptionText
        }
        val notificationManager: NotificationManager =
            applicationContext.getSystemService(Context.NOTIFICATION_SERVICE) as NotificationManager
        notificationManager.createNotificationChannel(channel)
    }



    override fun onBind(p0: Intent?): IBinder? {
        return null
    }

}
\end{lstlisting}

\begin{lstlisting}[language=Kotlin, caption=\leftline{AppViewModel}, label=lst:AppViewModel]
class AppViewModel: ViewModel()  {
    private val findList = MutableLiveData<List<Film>>()
    private val recommendList = MutableLiveData<List<Film>>()

    private val statusRecommendService = MutableLiveData(false)

    fun saveFindList(list : List<Film>){
        findList.value = list
    }
    fun getFindList() : List<Film>? {
        return findList.value
    }

    fun saveRecommendList(list : List<Film>){
        recommendList.value = list
    }
    fun getRecommendList() : List<Film>? {
        return recommendList.value
    }

    fun saveStatusRecommendService(flag : Boolean){
        statusRecommendService.value = flag
    }
    fun getStatusRecommendService() : Boolean? {
        return statusRecommendService.value
    }

    fun getObserveRecommendList() : LiveData<List<Film>> {
        return recommendList
    }

}
\end{lstlisting}