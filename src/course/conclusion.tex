\chapter*{ЗАКЛЮЧЕНИЕ}
\addcontentsline{toc}{chapter}{ЗАКЛЮЧЕНИЕ}
В ходе выполнения курсовой работы достигнуты следующие результаты:

\begin{itemize}
	\item Разработана логика работы сервиса рекомендательной системы;
	\item Разработана база данных для хранения информации о пользователях и
		продуктах;
	\item Разработан алгоритм индивидуальных рекомендаций;
	\item Разработано API для взаимодействия с сервисом;
	\item Разработано мобильное приложение для операционной системы
		Android;
	\item Разработана система контейнеризации для сервиса.
\end{itemize}

Программируя аппаратную компоненту мобильного телефона, в данной
работе, был освоен навык работы с программным интерфейсом приложения
андроид (Android API).\par
API это интерфейс, то есть абстракция, связывающая разные части
программного обеспечения между собой, например связь между
низкоуровневыми и высокоуровневыми программами.
Также его можно представить, как набор формальных определений, таких как классы, методы,
функции, модули и константы, которые могут использовать разработчики для
написания своего кода.\par
Суммируя, в данной работе было изучено устройство операционной
системы Android, ядро которой представляет собой ядро Linux, и работы с ней
через Android API.\par
Также был разработан дизайн для мобильного приложения.
Дизайн приложения предназначен для облегчения взаимодействия пользователя с ним
и определяет его узнаваемость среди прочих.
Еще, важно чтобы он следовал современным трендам в разработке.

