\begin{thebibliography}{00}
	\bibitem{1} Авдеев В.А. Организация ЭВМ и перифирия с демонстрацией имитационных моделей. – М.: ДМК, 2014. – 708 с.
	\bibitem{2} Антамошкин, О. А. Программная инженерия. Теория и практика. Учебник. M: НИЦ Инфра-М, 2012. – 368 с.
	\bibitem{3} Дейтел, Х., М. Операционные системы. Основы и принципы. Т. 1 – М.: Бином, 2016. - 1024 c
	\bibitem{4} Дейтел, Х., М. Операционные системы. Т. 2. Распределенные системы, сети, безопасность. М.: Бином, 2016. - 704 c
	\bibitem{5} Таненбаум, Э. Современные операционные системы / Э. Таненбаум. СПб.: Питер, 2013. - 1120 c.
	\bibitem{6} С.А.Орлов. Программная инженерия. Учебник для вузов. 5-е издание обновленное и дополненное.М: Издательский дом «Питер»,2017. 812 с.
	\bibitem{7} Дейв Тейлор. Сценарии командной оболочки. Linux, OS X и Unix. 2-е издание. Издательский дом «Питер»,2017.624 с.
\end{thebibliography}

