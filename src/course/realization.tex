\chapter{РАЗДЕЛ РЕАЛИЗАЦИЯ ПРОЕКТА}

\section{Реализация серверной части}

\subsection{База данных}
Для хранения данных была выбрана база данных PostgreSQL, и python библиотека для работы с ней SQLAlchemy.

В качестве тестовой базы данных была взята база данных кинопоиска, из которой были взяты фильмы с дополнительным коротким
описанием ~ 11000 фильмов.
Как основные характеристики были взяты:
\begin{itemize}
	\item Название;
	\item Фото;
	\item Длинное описание.
\end{itemize}

Остальные характеристики были помещены в отдельную таблицу, где каждая характеристика храниться в отдельной строке
Так же били созданы таблицы векторов, коротких описаний и лемматизированных коротких описаний, поскольку процессы по их
созданию довольно долгие и их не стоит делать каждый раз при запросе пользователя.

\subsection{Обработка данных}
Для обработки данных были использованы библиотеки:

\begin{itemize}
	\item pymystem3 - для лемматизации слов;
	\item nltk - для очисти текста от стоп слов;
	\item sentence\_transformers - для создания векторов текста.
\end{itemize}

При старте проекта проверяется флаг из конфига, если он True, то происходит парсинг и обработка данных из базы данных.

Обработка данных происходит в несколько этапов:

\begin{itemize}
	\item Парсинг данных из базы данных;
	\item Лемматизация коротких описаний;
	\item Очистка коротких описаний от стоп слов;
	\item Создание векторов коротких описаний.
\end{itemize}

\subsection{Рекомендательная система}

В ходе исследования было решено использовать ансамбль решающих деревьев, поскольку он показал лучшие результаты по
сравнению с косинусном расстоянием.

Для ансамбля была использована библиотека CatBoost, поскольку она показывает хорошие результаты на небольших данных,
при этом имея большую скорость работы.

При каждом запросе пользователя, происходит поиск в базе данных событий, которые пользователь оценил как понравившиеся
и не понравившиеся.
Из базы данных берутся векторы коротких описаний событий, а так же остальные данные для увеличения точности
предсказаний, для фильмов из кинописка это были:

\begin{itemize}
	\item жанры (в виде вектора)
	\item продолжительность
	\item год выпуска
	\item оценки критиков и пользователей
\end{itemize}

Данные подаются на вход CatBoostClassifier, модель обучается под данного пользователя, после предсказывает оценку
пользователя для каждого события.
После чего события сортируются по убыванию вероятности что понравится пользователю.
И возвращаться пользователю первые 5.

\subsection{Логика работы серверной части}
Все выполнено в SQLAlchemy в асинхронном режиме.

Работа с базой данных выполнена через обращения к статичным методам класса asyncHandler.

При каждом вызове функции, которая обращается к базе данных, происходит подключение к базе данных, через декоратор
@Session и после выполнения функции, происходит отключение от базы данных.

\subsection{Результаты рекомендательной сети}

Для оценки качества рекомендательной сети была использована метрика PRAUC(Precision-Recall AUC).

PRAUC - площадь под кривой точности и полноты.
Эта метрика показывает качество ранжирования положительных классов, Эта метрика подходит в рамках этой задачи,
поскольку мы ранжируем события, которые пользователь оценил бы как понравившиеся, а не понравившиеся нас не интересуют.

Для тестового датасета были выбраны фильмы про 1-2 мировую войну в качестве положительных событий, а в качестве
отрицательных событий были выбраны случайные фильмы не про войну.

Результаты тестирования:

1. Для тестового датасета были использованы только векторы коротких описаний

\begin{image}
	\includegrph{img1}
	\caption{Только векторы коротких описаний}
	\label{fig:test1}
\end{image}

\clearpage

2. Для тестового датасета были использованы векторы коротких описаний и остальные данные о фильмах

\begin{image}
	\includegrph{img2}
	\caption{Векторы коротких описаний и остальные данные о фильмах}
	\label{fig:test2}
\end{image}

Из-за того что модель при обучении использует случайные изначальные веса, точность 1 варианта не всегда меньше чем
во 2 варианте, но в среднем 2 вариант показывает лучшие результаты.

Точность на тестовой выборке в среднем 0.8 - 0.9, что является хорошим результатом.

\subsection{API}

Для работы с API была использована библиотека FastAPI, поскольку она показывает хорошие результаты по скорости работы
и имеет удобную документацию.

\begin{image}
	\includegrph{img3}
	\caption{Интерактивная документация к API}
	\label{fig:docsApi}
\end{image}

Реализованные методы:

lоgin - метод для авторизации пользователя, возвращает id пользователя, если он есть в базе данных, иначе возвращает
ошибку с кодом 404.
Входные данные - логин и пароль пользователя.

register - метод для регистрации пользователя, возвращает id пользователя, если такого пользователя нет в базе данных,
иначе возвращает ошибку с кодом 405.
Входные данные - логин и пароль пользователя.

find - метод для поиска продукта, возвращает список продуктов которые содержат в названии данную строчку.
Входные данные - строка для поиска.

rate -