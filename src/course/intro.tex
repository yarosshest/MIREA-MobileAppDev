\chapter*{ВВЕДЕНИЕ}
\addcontentsline{toc}{chapter}{ВВЕДЕНИЕ}

Цифровизация бизнеса является залогом его устойчивого развития и
конкурентоспособности.
Рекомендательные системы имеют ключевое
значение для цифровой трансформации.
Применение данного типа систем
является современной практикой в различных отраслях.
Системы активно применяются в маркетинге для анализа клиентского пути, персонализации
предложения, анализа контента и стратегии продвижения.
В медицине рекомендательные системы позволяют выявить возможные заболевания на
основе комплекса симптомов, подобрать лечение и пр.
Снабжение, логистика, список примеров можно расширять, поскольку рекомендательные
получили колоссальное распространение в последние годы.
Рекомендательные системы строятся на технологиях искусственного интеллекта.
Виды рекомендательных систем:
\begin{itemize}
    \item Контентная фильтрация; \par
    (content-based filtering) основывается на потребленном контенте.
    Например, пользователь зашел в онлайн-кинотеатр и купил боевик с Брюсом Уиллисом.
    Посмотрел его, поставил хороший балл.
    В следующий раз пользователю, скорее всего, посоветуют новый боевик с Брюсом Уиллисом.
    Рекомендательная система ориентируется на то, что пользователь посмотрел, и советует похожие фильмы.

    \item Коллаборативная фильтрация; \par
(collaborative filtering) учится на опыте других пользователей со схожими интересами.
    Представим, что есть Петя и есть Вася.
    Они оба любят мороженое, стейки и лимонад.
    Еще есть Андрей, который любит мороженое и лимонад.
    Рекомендательная система интернет-магазина с коллаборативной фильтрацией приходит к выводу, что тем, кто любит
    мороженое и лимонад, скорее всего, нравятся и стейки.
    Поэтому в следующий раз система порекомендует стейки и Андрею.
    \item Система, основанная на знаниях. \par
    В системе, основанной на знаниях (knowledge-based filtering), рекомендации строятся на основе экспертного мнения.
    В отличие от рекомендаций в предыдущих системах они чаще всего не персонализированные.
    Например, исследование компании показывает, что на севере России люди предпочитают есть рыбу, а на юге — курицу.
    Тогда в онлайн-магазине продуктов рекомендации могут отталкиваться от места жительства покупателя.
    \item Гибридная фильтрация. \par
    В гибридной системе (hybrid filtering) все перемешано: предыдущие виды рекомендаций могут работать вместе и
    подключаться в разных последовательностях.
    Например, параллельное подключение рекомендаций. \par
    В спортивном магазине есть пользователь, который занимается теннисом, живет на юге России и часто проводит время в
    разделе с товарами для походов.
    Алгоритм уже знает, что теннисисты сейчас увлекаются паддл-теннисом, а на юге пользователи чаще ездят на море летом
    и ходят в походы осенью.
    Кроме рекомендаций, основанных на прошлых покупках, система порекомендует ракетки для паддл-тенниса и палатку или
    зонтик от солнца в зависимости от сезона. \par
    Или последовательное: в музыкальном сервисе появляется новый пользователь, про которого еще ничего неизвестно.
    Система предлагает выбрать любимых исполнителей — это точка отсчета.
    Теперь алгоритм рекомендаций может обучаться.
    Например, он знает, что любители Тейлор Свифт часто слушают Гарри Стайлза, а слушателям джаза обычно по нраву блюз.
\end{itemize}


У большинства мобильных приложений нет рекомендательной системы либо это система, основанная на знаниях,
которая не подстраивается под пользователя.
За исключением больших интернет-магазинов.
Автоматизированная система «Рекомендательная система» будет предназначена для решения проблемы отсутствия
рекомендательной системы у небольшого и среднего бизнеса, без инвестиций в покупку личных данных пользователей.
