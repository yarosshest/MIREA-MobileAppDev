\graphicspath{{./img}}
\chapter{РАЗДЕЛ АНАЛИЗА И ИССЛЕДОВАНИЙ}
\section{Анализ предметной области}
Были проанализированны существующие системы рекомендаций, а также существующие алгоритмы рекомендаций.
В результате были выделены основные преимущества и недостатки существующих систем,
а также были выделены основные принципы, которыми руководствуются существующие алгоритмы рекомендаций.

\subsection{Существующие системы рекомендаций}
Существует множество систем рекомендаций, которые используются в различных областях.
В рамках данной работы были проанализированы системы рекомендаций, которые используют текстовое описание предметов.
В основном они рекомендуют предметы на основе исходного предмета, который пользователь выбрал.
Что не подходит для решения нашей задачи, так как мы хотим рекомендовать предметы на основе предпочтений пользователя.

Остальные же используют данные о других пользователях что не подходит для решения нашей задачи, так как мы хотим
рекомендовать предметы на основе предпочтений одного пользователя.

Однако, был выведен основной принцип, которым руководствуются существующие системы рекомендаций:
Текстовое описание предмета должно быть представлено в виде вектора, который содержит в себе информацию о предмете.
Текст лемматизируется, удаляются стоп-слова, и на основе полученных данных строится вектор описания предмета.
После вектор описания предмета сравнивается с векторами описания других предметов, и на основе этого строится
рейтинг предметов.

\subsection{Анализ реализации системы рекомендаций}
В ходе экспериментов было выяснено, что для реализации системы рекомендаций необходимо использовать ансамбль
решающих деревьев и вектоизированное описание.
Решающие деревья позволяют строить рекомендации на основе оценок пользователя, на обучение решающего дерева
необходимо подать 2 класса оценок: 0 и 1, где 0 означает, что пользователю не понравился предмет, а 1 означает,
что пользователю понравился предмет.

После обучения решающего дерева, можно получить предсказания для новых данных, которые будут содержать в себе класс
оценки, которую предсказало решающее дерево и вероятность этого класса.
Вероятность класса оценки позволяет ранжировать предметы, которые рекомендует система рекомендаций.

\subsection{Анализ архетектуры}
В ходе анализа архитектуры было выяснено, что для реализации системы рекомендаций необходимо использовать.
В качестве языка программирования для реализации системы рекомендаций был выбран язык программирования Python.

Базу данных для хранения данных о предметах, пользователях и оценках, в качестве базы данных была выбрана PostgreSQL.
А работа с ней через ORM библиотеку SQLAlchemy для языка программирования Python.

Для реализации API для взаимодействия с системой рекомендаций была выбрана библиотека FastAPI.

В качестве системы развертывания была выбрана система Docker.

Для общения с API со стороны клиента была выбрана библиотека Retrofit для языка программирования Kotlin.

\section{Дизайн мобильного приложения}
В данной курсовой работе необходимо продумать дизайн будущего мобильного приложения.
По теме курсовой работы разработано приложения для просмотра предметов и их оценке.

\subsection{Вход}
На данном экране пользователь может войти с помощью логина-пароля или перейти на экран регистрации.

\img{Login}{Активити входа}{0.5}{analis:Login}

\subsection{Вход}
На данном экране пользователь может зарегистрироваться.

\img{Register}{Активити регистрации}{0.5}{analis:Register}

\subsection{Посик}
На данном экране пользователь может найти фильмы по их названию.
И далее перейти просмотру информации фильма, и его оценки.

\img{FindFragment}{Фрагмент поиска}{0.5}{analis:FindFragment}

\subsection{Рекомендации}

На данном экране пользователь может просмотреть рекомендуемые ему фильмы.
И далее перейти просмотру информации фильма, и его оценки.

\img{RecommendationFragment}{Фрагмент рекомендаций}{0.5}{analysis:RecommendationFragment}

\subsection{Просмотр фильма}

На данном экране пользователь может посмотреть на постер фильма, прочитать его описание и оценить фильм.

\img{ProductFragment}{Фрагмент продукта}{0.5}{analysis:ProductFragment}

\subsection{Таймер}

На данном экране пользователь может поставить таймер контроля времени, которое он хочет выделить на просмотр фильма.
Пользователь может поставить время, остановить и запустить таймер.

\img{TimerFragment}{Фрагмент таймера}{0.5}{analysis:TimerFragment}

\img{TimerFragmentSetup}{Фрагмент таймера, настройка веремени}{0.5}{analysis:TimerFragmentSetup}